\documentclass[11pt,twoside,a4paper]{book}  
% definice dokumentu
\usepackage[czech, english]{babel}
\usepackage[T1]{fontenc} 				% pouzije EC fonty 
\usepackage[utf8]{inputenc} 			% utf8 kódování vstupu 
\usepackage[square, numbers]{natbib}	% sazba pouzite literatury
\usepackage{indentfirst} 				% 1. odstavec jako v~cestine, pro práci v~aj možno zakomentovat
\usepackage{fancyhdr}					% tisk hlaviček a~patiček stránek
\usepackage{nomencl} 					% umožňuje snadno definovat zkratky a~jejich seznam

%%%%%%%%%%%%%%%%%%%%%%%%%%%%%%%%%%%%%%%%%%%%%%%%%%%%%%%%%%%%%%%
% informace o práci
\newcommand\WorkTitle{Software pro distribuované řízení a vyčítání dat ze sítě částicových pixelových detektorů Timepix3} % název
\newcommand\FirstandFamilyName{Bc. Jakub Begera} % autor
\newcommand\Supervisor{Ing. Štěpán Polanský} % vedoucí

\newcommand\TypeOfWork{Diplomová práce}	% typ práce [Diplomová práce | Bakalářská práce | Bachelor's Project | Master's Thesis ]	

% Nastavte následují podle vašeho oboru a~programu (pomoc hledejte na http://www.fel.cvut.cz/cz/education/bk/prehled.html)								
\newcommand\StudProgram{Otevřená informatika} % program
\newcommand\StudBranch{Softwarové inženýrství} % obor

%%%%%%%%%%%%%%%%%%%%%%%%%%%%%%%%%%%%%%%%%%%%%%%%%%%%%%%%%%%%%%%
% minimální importy
\usepackage{graphicx}					% pro vkládání obrázků
\usepackage{k336_thesis_macros} 		% specialni makra pro formatovani DP a~BP
\usepackage[
pdftitle={\WorkTitle},				% nastaví v~informacích o pdf název
pdfauthor={\FirstandFamilyName},	% nastaví v~informacích o pdf autora
colorlinks=true,					% před tiskem doporučujeme nastavit na false, aby odkazy a~url nebyly šedé při ČB tisku
breaklinks=true,
urlcolor=red,
citecolor=blue,
linkcolor=blue,
unicode=true,
]
{hyperref}								% pro zobrazování "prokliknutelných" linků 

% rozšiřující importy
\usepackage{listings} 			%slouží pro tisk zdrojových kódů se syntax higlighting
\usepackage{algorithmicx} 		%slouží pro zápis algoritmů
\usepackage{algpseudocode} 		%slouží pro výpis pseudokódu
\usepackage{inconsolata}		%monospace font
\usepackage{soul}				%zvýrazňovač
\usepackage[final]{pdfpages}    %PDF include
\usepackage{subcaption}			%pro reference na subfigures 
\usepackage{enumitem}
\usepackage{bytefield}
\usepackage{pgf-umlsd}			%sekvenční UML diagramy
\usepackage{tikz}		        %TeX obrázky
\usetikzlibrary{trees,calc,positioning,shapes.geometric,matrix,arrows,decorations.pathreplacing,arrows.meta,decorations.markings,math}
\usepackage{pgfplots}

\usepackage{minted}				% slouží k~Syntax Highlighting
\usepackage{amsmath}

\usepackage{forest}
\definecolor{folderbg}{RGB}{124,166,198}
\definecolor{folderborder}{RGB}{110,144,169}
\definecolor{cvut_color}{RGB}{3,46,85}
\definecolor{blue_ligh}{RGB}{0,118,203}

\def\Size{4pt}
\tikzset{
  folder/.pic={
    \filldraw[draw=folderborder,top color=folderbg!50,bottom color=folderbg]
      (-1.05*\Size,0.2\Size+5pt) rectangle ++(.75*\Size,-0.2\Size-5pt);  
    \filldraw[draw=folderborder,top color=folderbg!50,bottom color=folderbg]
      (-1.15*\Size,-\Size) rectangle (1.15*\Size,\Size);
  }
}

%%%%%%%%%%%%%%%%%%%%%%%%%%%%%%%%%%%%%%%%%%%%%%%%%%%%%%%%%%%%%%%
% příkazy šablony
\makenomenclature								% při překladu zajistí vytvoření pracovního souboru se seznamem zkratek 
% zkratky CERN + ČVUT
\nomenclature{CERN}{Evropská organizace pro jaderný výzkum (Originální název: \textit{Conseil Européen pour la Recherche Nucléaire}), se sídlem v Ženevě, ve Švýcarsku.}
\nomenclature{ATLAS}{A Toroidal LHC Apparatus}
\nomenclature{LHC}{Large Hadron Collider}
\nomenclature{LS}{Long Shutdown - dlouhodobá technologická přestávka LHC}
\nomenclature{DCS}{Detector Control Systems}
\nomenclature{DSS}{Detector Safety System}
\nomenclature{DAQ}{Data Acquisition System}
\nomenclature{ÚTEF}{Ústav technické a experimentální fyziky}
\nomenclature{SATRAM}{Space Application of Timepix based Radiation Monitor}

% detektory
\nomenclature{TPX}{Timepix}
\nomenclature{MPX}{Medipix}
\nomenclature{TOT}{Time Over Treshold - mód detektoru (viz \ref{det:mod})}
\nomenclature{ToA}{Time of Arrival - mód detektoru (viz \ref{det:mod})}
\nomenclature{ASIC}{Application-specific Integrated Circuit}
\nomenclature{FPGA}{Field Programmable Gate Array}
\nomenclature{DAC}{Digitálně analogový převodník}
\nomenclature{ADC}{Analogově digitální převodník}
\nomenclature{CMOS}{Complementary Metal Oxide Semiconductor}
\nomenclature{CSM}{Charge Summing Mode}
\nomenclature{FITPix}{Fast Interface for Timepix Pixel Detectors}
\nomenclature{LVDS}{Low-voltage differential signaling}
\nomenclature{PCC}{Photon Counting Chip}
\nomenclature{ISO}{International Organization for Standardization}
\nomenclature{OSI}{Open Systems Interconnection model}

% software & hardware & elektronika
\nomenclature{HTTPS}{Hypertext Transfer Protocol Secure}
\nomenclature{HTTP}{Hypertext Transfer Protocol}
\nomenclature{API}{Application Programming Interface}
\nomenclature{REST}{Representational State Transfer}
\nomenclature{JSON}{JavaScript Object Notation}
\nomenclature{SQL}{Structured Query Language}
\nomenclature{URL}{Uniform Resource Locator}
\nomenclature{b}{bite}
\nomenclature{B}{byte}
\nomenclature{PC}{Personal Computer}
\nomenclature{HW}{Hardware}
\nomenclature{SW}{Software}
\nomenclature{USB}{Universal Serial Bus}
\nomenclature{PN}{Přechod polovodiče typu P a polovodiče typu N}
\nomenclature{TCP}{Transmission Control Protocol}
\nomenclature{IP}{Internet Protocol}
\nomenclature{SSH}{Secure Shell}
\nomenclature{PCB}{Printed Circuit Board}
\nomenclature{SPI}{Serial Peripheral Interface}
\nomenclature{GPIO}{General Purpose Input/Output}
\nomenclature{UDP}{User Datagram Protocol}
\nomenclature{SFTP}{Secure File Transfer Protocol}
\nomenclature{HDFS}{Hadoop Distributed File System}
\nomenclature{DOM}{Document Object Model}
\nomenclature{DSL}{Domain-specific language}
\nomenclature{J2EE}{Java Enterprise Edition}
\nomenclature{JPA}{Java Persistent API}
\nomenclature{JAR}{Java Archive}
\nomenclature{WAR}{Web Application Archive}
\nomenclature{BSON}{Binární forma JSON}
\nomenclature{JSON}{JavaScript Object Notation}
\nomenclature{REST}{Representational State Transfer}
\nomenclature{ID}{Identifikátor}
\nomenclature{LOB}{Large Object}
\nomenclature{CRUD}{Create, read, update and delete}
\nomenclature{HTML}{Hypertext Markup Language}
\nomenclature{MVC}{Model View Controller}
\nomenclature{ACQ}{Datagram typu acknowledge}
\nomenclature{ASCII}{American Standard Code for Information Interchange}
\nomenclature{UTC}{Koordinovaný světový čas (anglicky Coordinated Universal Time)}
\nomenclature{ORM}{Object-relational mapping}
\nomenclature{JRE}{Java Runtime Environment}
\nomenclature{CSS}{Cascading Style Sheets}
\nomenclature{JS}{JavaScript}
\nomenclature{IaaS}{Infrastructure as a Service}
\nomenclature{VCPU}{Virtuální procesor}

% matematické zkratky
\nomenclature{FWHM}{Full width at half maximum}

% fyzikální a chemické zkratky
\nomenclature{eV}{elektronvolt}
\nomenclature{GaAs}{Arsenid gallitý}
\nomenclature{CdTe}{Cadmium telluride}
\nomenclature{Si}{Křemík}
\nomenclature{In}{Indium}
\nomenclature{Fe}{Železo}
\nomenclature{Cu}{Hliník}


\let\oldUrl\url									% url adresy budou zobrazeny: <url> 
\renewcommand\url[1]{<\texttt{\oldUrl{#1}}>}
%\renewcommand\lstlistingname{Výpis kódu} % předefinování nadpisu
%\renewcommand\lstlistlistingname{Výpisy kódu} % předefinování nadpisu v~obsahu
%\renewcommand{\lstlistingname}{Algorithm}% Listing -> Algorithm
%\renewcommand{\lstlistlistingname}{Listdassad of \lstlistingname s}% List of Listings -> List
%%%%%%%%%%%%%%%%%%%%%%%%%%%%%%%%%%%%%%%%%%%%%%%%%%%%%%%%%%%%%%%
% vaše vlastní příkazy
\newcommand*{\nomExpl}[2]{#2 (#1)\nomenclature{#1}{#2}} % usnadňuje zápis zkratek : Slova ke Zkrácení (SZ)
\newcommand*{\nom}[2]{#1\nomenclature{#1}{#2}} % usnadňuje zápis zkratek : SZ
\newcommand*{\todo}{\hl{\textbf{TODO}}} % pro poznámky
\newcommand{\addbibresource}[1]{}
\newcommand*{\footnoteUrl}[1]{\footnote{\url{#1}}}
\newcommand{\unit}[2]{$#1~\ensuremath{\mathrm{#2}}$} % pro jednotky

% custom padding
\def\changemargin#1#2{\list{}{\rightmargin#2\leftmargin#1}\item[]}
\let\endchangemargin=\endlist 

\usepackage{caption}
\DeclareCaptionType{code}[Zdrojový kód][Seznam zdrojových kódů] 
%%%%%%%%%%%%%%%%%%%%%%%%%%%%%%%%%%%%%%%%%%%%%%%%%%%%%%%%%%%%%%%
% vlastní dokument
%%%%%%%%%%%%%%%%%%%%%%%%%%%%%%%%%%%%%%%%%%%%%%%%%%%%%%%%%%%%%%%
\begin{document}

	%%%%%%%%%%%%%%%%%%%%%%%%%% 
	% nastavení jazyka, kterým je práce psána
	\selectlanguage{czech}	% podle jazyka práce nastavte na [czech | english]
	\translate				% nastaví české nebo anglické popisy (např. katedra -> department); viz k336_thesis_macros


	%%%%%%%%%%%%%%%%%%%%%%%%%%    
	% Titulni stranka / Title page 
	\coverpagestarts

	%%%%%%%%%%%%%%%%%%%%%%%%%%    
	% Zadani / Assignment
	\newpage~
	\includepdf[pages=1,pagecommand={},offset=0cm -3cm]{zadani.pdf}
	\includepdf[pages=2,pagecommand={},offset=0cm -3cm]{zadani.pdf}
	\newpage

	%%%%%%%%%%%%%%%%%%%%%%%%%%%    
	% Poděkovani / Acknowledgements 
	\acknowledgements
	\noindent
	Na tomto místě bych rád poděkoval svým kolegům z ÚTEF ČVUT v Praze za poskytnutí know-how a za podporu při realizaci této práce, zejména Ing. Petrovi Burianovi, Ph.D. za rady a trpělivost při implementaci komunikačního protokolu vyčítacího rozhraní \textit{Katherine}. Také bych rád poděkoval vedoucímu této práce Ing. Štěpánu Polanskému za předané zkušenosti, ochotu a čas, který mi věnoval. Dále bych rád poděkoval M. Sc. Saúl Calderón Ramírez za vedení této práce během mého studijního pobytu na \textit{Instituto Tecnológico de Costa Rica} a za jeho cenné podněty. Mé poděkování také patří mé rodině a přátelům, kteří mě během studia podporovali. Bez nich by tato práce nikdy nemohla vzniknout.


	%%%%%%%%%%%%%%%%%%%%%%%%%%%   
	% Prohlášení / Declaration

	\declaration{V~Praze dne 8.\,1.\,2019}


	%%%%%%%%%%%%%%%%%%%%%%%%%%%%    
	% Abstrakt / Abstract 
 
	\abstractpage
	The aim of this master thesis is to design and implement Pixnet software -- modular distributed system for control and data readout from network of particle pixel detectors. System consists of two main components -- handler and master. Whereas handler is used for operation of subset of detectors and master is used for controlling of handlers and for providing API for controlling of whole system as well. Primary consumer of this API is a \textit{single-page} web application but it can be consumed by 3\textsuperscript{rd} party systems as well. Scalability of system is realized by adding handlers instances and distributing detectors across them.

	The presented system has a modular architecture -- instance of detector is given by his communication and data module, which has to implement defined interfaces. Thus, the system allows to operate heterogenous networks of detectors in homogenous way. Furthermore, the presented system includes a module implementation supporting \textit{Timepix3} detector, operated by \textit{Katherine} readout interface.
	
	For development and testing purposes, an emulator of the \textit{Katherine} interface has beed developed. Emulator implements subsets of commands which are needed to basic configuration of detector and data acquisition setup.

	\vspace*{2cm}
	\begin{description}
		\item[Key words:] Timepix3, ATLAS-TPX, Katherine, distributed control, ionizing radiation, pixel detector, Kotlin, detectors network, modular architecture, data acquisition software
	\end{description}

	\abstractpagecz
	Tato diplomová práce ze zabývá návrhem a implementací software Pixnet -- modulárním distribuovaným systémem pro řízení a vyčítání dat ze sítě částicových pixelových detektorů. Systém se skládá ze dvou hlavních komponent, kterými jsou handler a master. Handler slouží pro operování podmnožiny jemu přiřazených detektorů a master k řízení handlerů a poskytování API pro řízení celého systému. Primárním konzumentem tohoto API je webová \textit{single-page} aplikace, ale mohou jím být i systémy třetích stran. Škálování systému je realizováno přidáváním instancí handlerů a distribucí detektorů mezi jejich instancemi.
	
	Navržený systém je modulární -- instance detektoru je reprezentována jeho komunikačním a datovým modulem, které musí implementovat definované rozhraní. Systém tedy umožňuje operování heterogenní sítě detektorů homogenním způsobem. V rámci této práce byl implementován komunikační a datový modul, které umožňují podporu detektoru \textit{Timepix3}, komunikujícího prostřednictvím vyčítacího rozhraní \textit{Katherine}.

	Pro účely vývoje a testování byl v rámci této práce rovněž vyvinut emulátor vyčítacího rozhraní \textit{Katherine}, který implementuje všechny příkazy, které jsou důležité pro konfiguraci a spuštění akvizice dat.

	\vspace*{2cm}
	\begin{description}
		\item[Klíčová slova:] Timepix3, ATLAS-TPX, Katherine, distribuované řízení, ionizující záření, pixelový detektor, Kotlin, detektorová síť, modulární architektura, software pro akvizici dat
	\end{description}


	%%%%%%%%%%%%%%%%%%%%%%%%%%    
	% obsahy a~seznamy
	\tableofcontents		% Obsah
	\listoffigures			% Seznam obrázků
	\listoftables			% Seznam tabulek
	\listofcodes			% Seznam zdrojových kódů	

	%%%%%%%%%%%%%%%%%%%%%%%%%% 
	% začátek textu  
	\mainbodystarts

\addbibresource{reference.bib}

\chapter{Úvod}\label{chap01} 
\todo

\section{Motivace}
\todo

\section{Timepix3 detektor}
Hybridní částicový pixelový detektor Timepix3\cite{timepix3} je nástupcem detektoru Timepix\cite{timepix} a je vyvíjen v rámci Medipix\footnoteUrl{http://medipix.web.cern.ch/} kolaborace v CERN, mezi jejíž členy patří od roku 1999 i ÚTEF ČVUT v Praze.

\begin{figure}[h]
    \begin{center}
        \begin{subfigure}{7.0cm}
            \includegraphics[width=7cm]{figures/timepix3.jpg}    
            \caption{Fotografie Timepix3 detektoru \cite{medipix_from_medical_img_to_space}.}
        \end{subfigure}
        \hspace{0.1cm}
        \begin{subfigure}{7.0cm}
            \includegraphics[width=7cm]{figures/timepix_data_satram.png}    
            \caption{Ukázka dat z detektoru Timepix, naměřených zařízením \textit{SATRAM}, které od roku 2013 obíhá Zemi na heliosynchronní kruhové polární dráze ve výšce \unit{820}{km} \cite{PlatkevicDisertace}.}
        \end{subfigure}
	\end{center}
    \caption{Detektor Timepix3 a vizualizace naměřeného vzorku dat.}
	\label{fig:master:frontend:detector_detail}
\end{figure}

Detektor se skládá z matice $256\times256$ nezávislých pixelů, každý o hraně $55~\mu m$. 
Jednotlivé pixely se skládají z citlivého polovodičového senzoru (nejčastěji \textit{Si}, nebo \textit{GaAs}) a vyčítací CMOS elektroniky (čítače, komparátory apod.). Princip funkce detektoru lze přirovnat digitálnímu fotoaparátu. Podobně jako u digitálního fotoaparátu, začátek a konec akvizice dat je řízen uzávěrkou (tzv. \textit{shutter signál}). Po tuto dobu pak citlivý polovodičový objem detektoru zaznamenává interakce s nabitými částicemi a dále je zpracovává dle nastaveného modu. V kapitole \ref{chap:detectors} bude na příklad popsán \textit{Time-Over-Threshold} mód, kde hodnota čítače pixelu na konci akvizice odpovídá deponované energii interagovaných částic s daným pixelem (mezi energií a TOT je nelineární závislost, která je dána fyzikálními vlastnostmi každého pixelu a je předmětem energetické kalibrace detektoru \cite{Jakubek2011S262}). 

Timepix3 detektor přináší oproti svému předchůdci několik výhod. Je schopný operovat i v kontinuálním módu, ve kterém je každý pixel detektoru schopný detekovanou událost ihned zpracovat, nezávisle na ostatních pixelech. Tím se téměř odstraňuje mrtvá doba detektoru, zvyšuje detekční účinnost, ale i zvyšuje datový tok z detektoru, jehož maximální teoretická hodnota je až $5.12 Gb/s$.

\section{Struktura práce}
\todo
\addbibresource{reference.bib}

\chapter{Úvod do částicových pixelových detektorů}\label{chap:detectors}

\section{Hardwarová architektura}

\addbibresource{reference.bib}

\chapter{Návrh architektury}\label{chap:arch}
V této kapitole bude čtenář seznámen s návrhem a koncepcí softwarového systému \textbf{Pixnet} - software pro distribuované řízení sítě částicových pixelových detektorů, který byl navržen a implementován v rámci této práce. V této kapitole bude popsána motivace pro vznik tohoto systému a budou představeny jednotlivé komponenty systému a jejich vzájemné interakce. Pro detailnější popis návrhu a implementace komponent viz kapitoly \todo (přidat ref na handler až master).

%********************************************************************************
% Motivace
%********************************************************************************
\section{Motivace}
Hlavní motivací pro vznik tohoto systému je fakt, že moderní částicové pixelové detektory jsou schopné generovat vysoký datový tok, na příklad \textit{Timepix3} má teoretické maximum \unit{5,12}{Gb/s} (viz \ref{chap:detectors:medipix_overview:timepix3}), takže nedistribuovaný systém, který by operoval na jedné instanci, by nebyl schopný zpracovat datový tok, který síť o vice detektorech je schopná generovat. 

Zde je možné namítnout, že každý systém je možné škálovat vertikálně\footnote{Škálováním v kontextu počítačových systému rozumíme změnu vlastností daného systému za účelem zvýšení, nebo snížení jeho výpočetního výkonu (ev. jiného sledovaného parametru). Zatímco u vertikálního škálování měníme vlastnosti jednoho uzlu systému (na příklad přidáváním procesorů, pamětí, kapacity úložiště apod.), u horizontálního škálování přidáváme jednotlivé uzly - samostatné  jednotky (na př. počítače). Pro úplnost je třeba doplnit že vertikální škálování má své omezení z hlediska použitého hardware, u horizontálního škálování žádná taková omezení nejsou.}. Zatímco cena škálování horizontálně škálovatelného systému je lineární závislost výpočetního výkonu na ceně, u vertikálně škálovatelného systému tato závislost roste exponenciálně. Jelikož vertikální škálování takového systému je  vysoce neefektivní, nebude dále uvažováno a tato práce se bude věnovat jenom návrhu a implementaci horizontálně škálovatelného řešení.

 Další motivací pro vytvoření tohoto systému je možnost řízení heterogenní sítě detektorů homogenním způsobem. Heterogenní sítí detektorů rozumíme takovou síť, ve které jsou detektory různých typů (na příklad \textit{Timepix}, \textit{Timepix3} apod, viz \ref{chap:detectors:medipix_overview}), komunikující různými komunikačními protokoly prostřednictvím různých vyčítacích rozhraních (na příklad \textit{Katherine}, \textit{ATLAS Pix} apod, viz \ref{chap:detectors:readouts}). V další části textu bude detailně popsána navržená a implementovaná modulová architektura, která výše zmíněné umožňuje.

 Pro potřeby experimentu \textit{ATLAS TPX} byl již vyvinut software \cite{atlastpx_sw,BegeraBcThesis2016} pro řízení sítě detektorů \textit{Timepix} \ref{chap:detectors:medipix_overview:timepix}, prostřednictvím vyčítacího rozhraní \textit{ATLAS Pix} \ref{chap:detectors:readouts:atlaspix}. Software však nevyhovuje požadavkům zmíněných výše:
 \begin{enumerate}[label=(\roman*)]
     \item \textbf{Škálovatelnost} - systém je navržen bez možnosti horizontálního škálování. Všechny detektory sítě jsou řízeny z jednoho uzlu a všechna vygenerovaná data jsou jím zpracovávány. Možnost použití pouze jednoho uzlu představuje nejslabší článek systému, který nemůže být použit pro řízení a vyčítání dat z větší sítě detektorů.
     \item \textbf{Modularita} - systém implementuje pouze komunikační protokol vyčítacího rozhraní \textit{ATLAS Pix} \ref{chap:detectors:readouts:atlaspix}. Přidání podpory nového vyčítacího rozhraní představuje významnou modifikaci architektury systému a pro nasazení nové verze je nutná odstávka celého systému.
 \end{enumerate}
Pro potřeby modernizace sítě \textit{ATLAS TPX} (za použití detektorů \textit{Timepix3}) bylo rozhodnuto o vývoji software, který bude navržen a implementován tak, aby požadavky na škálovatelnost a modularitu byly zajištěny.
\addbibresource{reference.bib}

\chapter{Handler}\label{chap:handler}

V této kapitole bude čtenář podrobně seznámen s handlerem - komponentou pro distribuované řízení detektorů. Jak již bylo zmíněno v kapitole \ref{chap:arch}, handler je komponenta zajišťující komunikaci a vyčítání dat z detektorů (viz obr. \ref{fig:handler:overview}) skrze poskytnuté implementace komunikačního a datového interface.
  
Handler implementuje \textit{Spring framework} (viz \ref{chap:arch:technologie:spring}), aby mohl poskytovat \texttt{REST API} pro management detektorů a pro poskytování stavových informací o své instanci. Handler dále poskytuje jednoduché webové rozhraní s přehledem připojených detektorů.

Na obrázku \ref{fig:handler:overview} je znázorněn příklad instance handleru z pohledu vstupu (pět přiřazených detektorů) a pohledu výstupu (poskytované \texttt{REST API} a webové rozhraní.)

\begin{figure}[bh]
	\begin{center}
		\vspace*{1cm}
		\includegraphics[width=14cm]{figures/handler_overview.pdf}
		\caption{Pixnet - handler: příklad instance handleru s pěti připojenými detektory, poskytujícího REST API pro své řízení a webové uživatelské rozhraní s přehledem připojených detektorů.}
		\label{fig:handler:overview}
	\end{center}
\end{figure}

\newpage

%********************************************************************************
% Vrstvy softwarové architektury
%********************************************************************************
\section{Vrstvy softwarové architektury}\label{chap:handler:architecture}
Tato podkapitola je věnována softwarové architektuře handleru. Budou zde představeny jednotlivé její vrstvy, tj. od vrstvy pro komunikaci s detektorem, management detektorů, Spring implementaci, až po výstupní vrstvu poskytující \texttt{REST API} a webové rozhraní - viz obr. \ref{fig:handler:arch}. Vybrané vrstvy budou popsány detailněji v dalších podkapitolách.

\begin{figure}[th]
	\begin{center}
		\vspace*{0.4cm}
		\includegraphics[width=14cm]{figures/handler_architecture.pdf}
		\caption{Pixnet - handler: softwarová architektura s vrstvami pro (i) rozhraní detektoru (\textit{Detector layer}), (ii) management detektorů (\textit{Detectors management layer}),(iii) Spring vrstvu (\textit{Spring layer}) a (iv) výstupní vrstvou pro \texttt{REST API} a webové uživatelské rozhraní.}
		\label{fig:handler:arch}
	\end{center}
\end{figure}

\subsection{Detektorová vrstva}\label{chap:handler:detector_layer}
Detektorová vrstva se skládá ze dvou částí - externě poskytnuté implementace komunikačního a datového interface. Tyto tzv. moduly jsou zavedené až za běhu programu pomocí vyšších vrstev systému.

\subsubsection{Zavádění modulů}
Každý modul, obsahující implementaci komunikačního, nebo datového interface, musí pro své zavedení do systému splňovat následující kritéria:

\begin{itemize}
	\item Musí být zkompilován do Java archívu (\texttt{*.jar}) - současná verze implementace podporuje pouze moduly vyvinuté v jazyce Java (resp. také v jazyce Kotlin, zkompilovaného do Java \textit{bytecode}), v dalších verzích je plánováno přidání podpory pro \texttt{C++} a \texttt{Python}.
	\item Při vývoji modulu musí být použit datový model, který je součástí poskytovaných knihoven.
	\item Modul musí obsahovat implementaci jednoho z již zmíněních rozhraní. Navíc, celý název implementující třídy (vč. tzv. \texttt{package}) musí být uveden jako atribut s názvem \texttt{PluginImplClass} v manifestu \texttt{jar} archívu, viz zdrojový kód \ref{src:handler:manifest}.
\end{itemize}

\begin{minted}[
    frame=single,
	linenos,
	breaklines
  ]{html}
  Manifest-Version: 1.0
  PluginImplClass: cz.ctu.ieap.pixnet.handler.detector_communication_katherine.CommImpl  
\end{minted}
\begin{code}[h!]
\caption{Příklad obsahu souboru \texttt{MANIFEST.MF}, obsaženého v \texttt{jar} archívu modulu.}
\label{src:handler:manifest}
\end{code}


\subsubsection{Komunikační rozhraní}
Zdrojový kód \ref{src:handler:comm_intf} obsahuje interface, který komunikační modul detektoru musí implementovat.

Prvních pět metod (tj. řádek 3 až 7) má čistě informativní charakter a jejich implementace má čistě informativní charakter pro operátora systému.

Na řádcích 8 a 9 je setter a getter pro konfiguraci detektoru. Konfigurace je detektoru předána jako \texttt{String}\footnote{Datový typ obsahující textový řetězec} (setter) a musí být do něj serializovatelná (getter). Systém ale s konfigurací detektoru neumí pracovat (kromě komunikačního modulu detektoru, resp. implementace komunikačního interface), tudíž jejíž syntax není vynucována a její podoba je čistě v kompetenci poskytovatele komunikačního modulu. Avšak je doporučeno, aby zvolený formát byl strojově i lidsky čitelný, z důvodu jeho snazší editace\footnote{V dalších fázích implementace systému je plánováno přidání podpory editace konfigurace v rámci webového uživatelského rozhraní mastera.}. Takový formát může být na příklad \texttt{JSON}\footnote{Z angl. \textit{JavaScript Object Notation} (JavaScriptový objektový zápis).}, nebo \texttt{YAML}\footnoteUrl{http://yaml.org/}, který je použit pro serializace konfigurace komunikačního modulu Katherine (viz \todo přidat ref.).

\begin{minted}[
    frame=single,
	linenos,
	breaklines
  ]{kotlin}
interface DetectorComm {

  fun getDetectorType(): DetectorType
  fun getReadoutName(): String
  fun getSensorsCount(): Int
  fun getDetectorWidth(): Int
  fun getDetectorHeight(): Int
  fun setDetectorConfig(config: String)
  fun getDetectorConfig(): String?
  fun getSupportedValueCommands(): List<AbstractValueCommand>
  fun getSupportedExecutionCommands(): List<AbstractExecutionCommand>
  fun getAcceptedFilesKeys(): List<String>
  fun getDataFrameQueue(): BlockingQueue<AbstractDataFrame>
  
  fun isConnected(): Boolean
  fun connect(): Boolean
  fun disconnect(): Boolean
  
  fun executeSetValueCommand(commandID: Int, payload: ValuePayload)
  fun executeGetValueCommand(commandID: Int): ValuePayload
  fun executeExecutionCommand(commandID: Int, input: Map<String, ValuePayload>): Map<String,  ValuePayload>
  fun uploadFile(fileKey: String, file: ByteArray)
  fun setCallback(callback: Callback)
  
  interface Callback {
    val classLoader: ClassLoader?
  }

}
\end{minted}
\begin{code}[h!]
\caption{Komunikační interface detektoru, napsané v jazyce Kotlin (viz \ref{chap:arch:technologie:kotlin})).}
\label{src:handler:comm_intf}
\end{code}

Pro získání seznamu podporovaných \textit{value commands} (tj. příkazů pro operace s jednotlivými hodnotami detektoru) slouží metoda \texttt{getSupportedValueCommands()}, viz řádek 10. \texttt{AbstractValueCommand} má v modelu poskytované knihovny dvě implementace:
\begin{enumerate}[label=(\roman*)]
	\item \texttt{ValueCommand} obsahuje atributy pro daného příkazu, tj.:
	\begin{itemize}
		\item \textbf{id} - celočíselný unikátní identifikátor daného příkazu,
		\item \textbf{name} - název příkazu, resp. manipulované hodnoty detektoru,
		\item \textbf{valueUnit} - jednotka veličiny manipulované hodnoty detektoru (může nabývat hodnot z \textit{Enum} třídy \texttt{ValueUnit} poskytovaného modelu, např. \texttt{ValueUnit.VOLT} apod.),
		\item \textbf{accessType} - modifikátor přístupu manipulované hodnoty, který může nabývat těchto hodnot:
		\begin{description}
			\item[\texttt{SETTER}] pro takové hodnoty, které je možné pouze nastavovat,
			\item[\texttt{GETTER}] pro takové hodnoty, které je možné pouze číst a
			\item[\texttt{SETTER\_AND\_GETTER}] pro hodnoty, které je možné nastavovat i číst.
		\end{description}
		\item \textbf{valueModel} - datový model hodnoty daného příkazu, který definuje datový typ veličiny (podporovány jsou \texttt{Boolean}, \texttt{String}, \texttt{Integer}, \texttt{Long}, \texttt{Float} a \texttt{Double}) a omezení rozsahu hodnot. Model může být diskrétní (tzn. hodnota může nabývat jen nějaké z předem definovaných hodnot), nebo spojitý (hodnota může nabývat jakékoliv hodnoty ze zadaného intervalu).
	\end{itemize}
	Viz zdrojový kód \ref{src:handler:value_command} pro příklad definice \textit{ValueCommand} pro \textit{bias} (prahové napětí detektoru).

	\item \texttt{ValueCommandGroup} je třída pro seskupování příkazů podobného významu (např. \texttt{DAC} hodnoty detektoru) a obsahuje název skupiny příkazů a seznam jednotlivých\\\texttt{ValueCommands}.
\end{enumerate}

\begin{minted}[
    frame=single,
	linenos,
	breaklines
  ]{java}
valueCommands.add(ValueCommand(
    42, // id
    "Bias", // name
    ValueUnit.VOLT, // valueUnit
    SETTER_AND_GETTER, // accessType
    FloatValueModel(-300f, 300f) // valueModel
))
\end{minted}
\begin{code}[h!]
\caption{Příklad definice \textit{ValueCommand} detektoru pro příkaz s názvem \textit{"Bias"}, id 42, jednotkou Volt, modifikátorem přístupu \textit{Setter \& Getter} a reálným modelem hodnot, omezeným intervalem $<-300,300>$.}
\label{src:handler:value_command}
\end{code}

Pro vykonání \texttt{ValueCommand} je třeba implementovat metody \texttt{executeSetValueCommand()} a \texttt{executeGetValueCommand()}, viz řádky 19 a 20 komunikačního interface (zdrojový kód \ref{src:handler:comm_intf}). První metoda slouží pro nastavení hodnot a akceptuje ID příkazu a \texttt{valuePayload}, který obsahuje hodnotu jednoho ze šesti podporovaných datových typů, a je možné jej vykonat pouze pro příkazy, které mají \texttt{accessType}, umožňující zápis hodnot. Druhá metoda slouží pro čtení hodnot, jako vstupní parametr má ID příkazu a její návratový typ je \texttt{ValuePayload}, obsahující hodnotu čteného parametru.

Dalším typem příkazů je získání seznamu podporovaných \texttt{ExecutionCommands} (viz 11. řádek komunikačního interface \ref{src:handler:comm_intf}). Od \texttt{ValueCommands} se liší tím, že vstup a výstup není omezen stejnou veličinou (resp. jejím modelem hodnot) a ani jejich množstvím. Tento přístup tedy umožňuje definovat příkazy, které mají 0 až m vstupních hodnot a 0 až n výstupních hodnot. Obdobně jako \texttt{AbstractValueCommand}, i \texttt{AbstractExecutionCommands} má v poskytované knihovně dvě implementace:
\begin{enumerate}[label=(\roman*)]
	\item \texttt{ExecutionCommand} je definován obdobně jako \texttt{ValueCommand} - má své ID, jméno, ale také obsahuje seznamy rozšířených modelů hodnot - jeden vstupní a jeden výstupní. Rozšířený model hodnot obsahuje již zmíněný \texttt{valueModel}, dále pak jméno hodnoty, její ID (textový řetězec) a \texttt{valueUnit}.
	
	Zdrojový kód \ref{src:handler:execution_command} obsahuje příklad s \texttt{ExecutionCommand} pro nastavení akvizičního módu detektoru. Z příkladu je patrné, že příkaz akceptuje právě dvě vstupní hodnoty (akviziční mód a \textit{FastVCO} přepínač) a že žádné hodnoty nevrací.
	
	\item \texttt{ExecutionCommandGroup} je obdobou třídy \texttt{ValueCommandGroup} pro \texttt{ExecutionCommand}. Obsahuje název skupiny příkazů a jejich seznam.
\end{enumerate}

\begin{minted}[
    frame=single,
	linenos,
	breaklines
  ]{java}
executionCommands.add(ExecutionCommand(
  KatherineExecutionCommands.SET_ACQ_MODE.internalID, // ID (Int)
  "Set acquisition mode", // name
  // model vstupných hodnot
  arrayOf(
    ValueModelVerbose(
      "Acquisition mode", // název hodnoty
      ValueId.acq_mode.name, // ID hodnoty (String)
	  // celočíselný diskrétní model hodnot
	  IntValueModel(mapOf(
        Pair("ToA & ToT", 0),
        Pair("ToA", 1),
        Pair("Event & iToT", 2)
      )),
      ValueUnit.DIMENSIONLESS // jednotka hodnoty
    ), 
    ValueModelVerbose(
      "Fast vco enabled",
      ValueId.fast_vco_en.name,
      BooleanValueModel(mapOf(
        Pair("Enable", true),
        Pair("Disable", false)
      )),
      ValueUnit.DIMENSIONLESS
	)
  ),
  null // model výstupních hodnot
))
\end{minted}
\begin{code}[h!]
\caption{Příklad definice \textit{ExecutionCommand} pro nastavování akvizičního módu detektoru s vyčítacím rozhraním \textit{Katherine} (viz \ref{chap:detectors:readouts:katherine}). Z příkladu je patrné, že vstupní model je tvořen dvěma hodnotami a výstupní model je prázdný.}
\label{src:handler:execution_command}
\end{code}

Pro vykonání \texttt{ExecutionCommand} slouží metoda \texttt{executeExecutionCommand()}, viz řádek 21 komunikačního interface (zdrojový kód \ref{src:handler:comm_intf}). Metoda akceptuje ID příkazu a mapu (tzn. seznam párů klíč - hodnota) jednotlivých hodnot. Klíčem v mapě je ID parametru a hodnota je již výše zmíněný \texttt{ValuePayload}, obsahující hodnotu parametru. Výstupem volání metody je pak mapa výstupních parametrů příkazu.

Pro přípojení detektoru, odpojení detektoru a zjišťování stavu připojení slouží metody \texttt{connect()}, \texttt{disconnect()} a \texttt{isConnected()} (viz řádky 15 až 17 zdrojového kódu \ref{src:handler:comm_intf}). Aby bylo možné provést jakoukoliv operaci interagující s detektorem (tj. vykonávání příkazů a nahrávání souborů) je nutné, aby byl detektor připojen, resp. metoda \texttt{isConnected()} musí vracet \texttt{true}.

Do detektorů rodiny \textit{Medipix} je třeba před jejich použitím nahrát konfiguraci jednotlivých pixelů detektoru. Konfigurace je pole bytů, kde nastavení jednoho pixelu je serializováno do jednoho bytu. Z tohoto důvodu je třeba přidat podporu pro nahrávání velkých binárních objektů (tzv. \texttt{LOB}\footnote{Z angl. Large Object.}).
Pro nahrávání souborů do detektoru slouží příkaz \texttt{uploadFile()} (viz řádek 22 zdrojového kódu \ref{src:handler:comm_intf}), který akceptuje ID souboru a jeho binární reprezentaci. Pro poskytnutí seznamu podporovaných souborů, resp. jejich ID, je třeba implementovat metodu \texttt{getAcceptedFilesKeys()} (viz řádek 12 zdrojového kódu \ref{src:handler:comm_intf}).

V kapitole \ref{chap:arch:sw:detector} již byl popsán tok měřených dat v handleru, resp. od komunikačního k datovému modulu. Přenos dat je realizován asynchronní blokující frontou (viz obr. \ref{fig:handler:data_queue}), tj. frontou ke které může asynchronně přistupovat více vláken současně a zároveň čtení z fonty je implementováno jako blokující operace (tzn. že zablokuje čtecí vlákno, než bude ve frontě nějaký element v vyčtení). Jelikož komunikační modul je zodpovědný za vytvoření instance fronty, tak její implementace je zcela v kompetenci poskytovatele komunikačního modulu. Jedinou podmínkou je, aby fronta implementovala interface \texttt{BlockingQueue}\footnote{Z \textit{Java Collections Framework}.}. V další části textu (viz \ref{chap:katherine}) bude čtenář seznámen s implementací komunikačního modulu s implementací fronty pomocí \texttt{LinkedBlockingQueue}\footnote{Implementace \texttt{BlockingQueue} pomocí spojového seznamu. V této frontě jsou elementy řazení pomocí FIFO (first-in-first-out). Fronty založené na spojové struktuře umožňují (ve srovnání s frontami založenými na dynamickém poly) vyšší datový tok, zejména pro vícevláknové aplikace.}.
\begin{figure}[h]
	\begin{center}
		\vspace*{0.5cm}
		\includegraphics[width=14.5cm]{figures/handler_data_queue.pdf}
		\caption{Asynchronní blokující fronta naměřených dat s příkladem produkujících vláken v komunikačním modulu a přijímacím vláknu v datovém modulu.}
		\label{fig:handler:data_queue}
	\end{center}
\end{figure}

Poslední metodou komunikačního interface je metoda \texttt{setCallback()} (viz 23. řádek komunikačního interface (zdrojový kód \ref{src:handler:comm_intf})). Tato metoda je handlerem zavolána bezprostředně po inicializaci modulu a modul si může uložit referenci na předaný \texttt{callback} (instanci interface \texttt{Callback} z komunikačního interface). \texttt{Callback} slouží například k získání Java \texttt{ClassLoader}, kterým byl modul načten (používá se například pro parsing konfigurace).

\subsubsection{Datové rozhraní}


\subsection{Vrstva managementu detektorů}\label{chap:handler:detectors_layer}


\subsection{Spring vrstva}\label{chap:handler:spring}
\addbibresource{reference.bib}

\chapter{Vyčítací rozhraní Katherine a jeho implementace}\label{chap:katherine}
V této kapitole bude čtenář blíže seznámen s komunikačním protokolem vyčítacího rozhraní \textit{Katherine} (viz \ref{chap:detectors:readouts:katherine}) a bude zde také představena implementace komunikačního (viz \ref{src:handler:comm_intf} ) a datového (viz \ref{src:handler:data_intf}) interface, vyvinutých za účelem podpory \textit{Katherine} handlerem (viz \ref{chap:handler}). 

Pro účely vývoje a testování byl vyvinut emulátor \textit{Katherine}, který emuluje zařízení na síťové vrstvě, včetně podpory všech příkazů komunikačního protokolu relevantních k akvizici dat. Popis jeho návrhu a implementace bude také zahrnut v této kapitole.

Pro úplnost je třeba dodat, že komunikační a datový modul sdílejí stejnou podmnožinu datového modelu a část business-logiky, vázající se k manipulaci s měřenými daty. Tato společná \textit{code-base} je implementována v rámci separátního modulu, na kterém jsou závislé oba výše zmíněné moduly. Toto řešení umožňuje použití stejné vnitřní reprezentace dat, bez nutnosti jejich serializace a deserializace, což má pozitivní vliv na výkonost systému. Nutnou podmínkou ale je, aby oba moduly byly do systému zavedeny pomocí stejného \texttt{ClassLoader} objektu (z důvodu kompatibility přenášených model objektů). Tento mechanizmus je ale už implementován v handleru.

%********************************************************************************
% Komunikační protokol
%********************************************************************************
\section{Komunikační protokol}\label{chap:katherine:protocol}
Jak již bylo vysvětleno v kapitole \ref{chap:detectors:readouts:katherine:comm}, \textit{Katherine} podporuje dva provozní módy -- autonomní a manuální. Tato práce se zabývá pouze použitím manuálního módu, který umožňuje plnohodnotné řízení detektoru a umožňuje (ve srovnání s autonomním módem) efektivněji přenášet naměřená data, díky čemuž je možné maximalizovat tok výstupních dat detektoru.

Komunikace v rámci manuálního módu je založena na proprietárním komunikačním protokolu, který bude popsán v této podkapitole. Klientská část tohoto protokolu je implementována v komunikačním modulu handleru (viz \ref{chap:katherine:comm}) a jeho serverová část v \textit{Katherine} emulátoru (viz \ref{chap:katherine:emulator}).

Protokol je založen na posílání \texttt{UDP} datagramů pomocí dvou socketů\footnote{Označení pro koncový bod obousměrné komunikace mezi počítačovými programy v rámci počítačové sítě.} -- jednoho pro řídící příkazy a druhého pro měřená data. Využití \texttt{UDP} se nabízí především z důvodu implementovaného nepotvrzovaného spojení (na rozdíl od \texttt{TCP}), takže v případě špatného spojení nedochází k zahlcení a je možné dosáhnout většího datového toku. V dalších fázích vývoje je plánován přechod na \texttt{TCP} protokol pro řídící socket, protože objem jeho prostřednictvím přenášených dat je zanedbatelný a potvrzované doručení řídících příkazů by nemuselo být ošetřováno aplikační vrstvou.

%********************************************************************************
% Komunikační protokol -- Řídící příkazy
%********************************************************************************
\subsection{Řídící příkazy}\label{chap:katherine:protocol:control_commands}
Každý řídící příkaz je vždy inicializován klientem -- klient pošle 8-bytový datagram na předem definovaný komunikační port \textit{Katherine}, jehož struktura je znázorněna na obr. \ref{fig:katherine:protocol:comm_packet:request} a server (resp. \textit{Katherine}) vždy pošle odpověď ve formě datagramu (viz \ref{fig:katherine:protocol:comm_packet:response}), nebo jiných dat dle specifikace příkazu, na port klienta, ze kterého spojení bylo inicializováno\footnote{Doplnění této funkcionality bylo vyžádáno až v průběhu implementace komunikačního modulu a během psaní této práce je ještě ve fázi testování. V současné produkční verzi firmware \textit{Katherine} tedy posílá response datagram na staticky konfigurovaný port, což pro operování více \textit{Katherine} o stejné statické konfiguraci z jednoho handleru znamená konflikt portů.}. Response některých příkazů je jen potvrzení (tzv. \textit{Acknowledge}) ve formě ID příkazu a daty $0x00000000$ (dále jen ACQ). Bytové pořadí pro všechny datagramy je \textit{BigEndian}\footnote{Název pro označení endianity (bytového pořadí), kde na paměťové místo s nejnižší adresou je uložen nejvíce významný byte (MSB).}.

\begin{figure}[h]
	\begin{center}
		\begin{subfigure}{7.0cm}            
            \begin{bytefield}[endianness=big,bitwidth=0.25em]{64}
                \bitheader{63,48} \\
                \bitbox{16}{\textbf{ID}} & \bitbox{48}{Requst data (\unit{6}{B})}
            \end{bytefield}
			\caption{Request.}
			\label{fig:katherine:protocol:comm_packet:request}
		\end{subfigure}
		\hspace{0.1cm}
		\begin{subfigure}{7.0cm}            
            \begin{bytefield}[endianness=big,bitwidth=0.25em]{64}
                \bitheader{63,48} \\
                \bitbox{16}{\textbf{ID}} & \bitbox{48}{Response data (\unit{6}{B})}
            \end{bytefield}
			\caption{Response.}
			\label{fig:katherine:protocol:comm_packet:response}
		\end{subfigure}
	\end{center}
	\caption{Struktura řídícího datagramu.}
	\label{fig:katherine:protocol:comm_packet}
\end{figure}

Následuje výčet všech důležitých řídících příkazů, s jejich ID, názvem a stručným popisem.

\begin{description}
    \item[0x01 -- Set acq. time (LSB)] -- nastaví nejnižších 32 bitů akvizičního času v desítkách nanosekund.
    \\\textit{Request}: \texttt{data[31..0]} LSB akvizičního času.
    \\\textit{Response}: ACQ.

    \item[0x02 -- Set bias] -- nastaví \textit{bias} napětí detektoru.
    \\\textit{Request}: \texttt{data[39..32]} ID biasu, \texttt{data[31..0]} float\footnote{IEEE 754 standart pro reprezentaci čísel s plovoucí desetinou čárkou.} hodnota biasu.
    \\\textit{Response}: ACQ.
    
    \item[0x03 -- Acquisition Start] -- spuštění akvizice dat.
    \\\textit{Request}: \texttt{data[0]} akviziční mód (0 = sekvenční, 1 = data-driven).
    \\\textit{Response}: ACQ.
    
    \item[0x04 -- Internal DAC Settings] -- nastavení hodnoty DAC (digitálně analogového převodníku) detektoru. Přehled jednotlivých dat je uveden v příloze \ref{chap:app:katherine:dacs}.
    \\\textit{Request}: \texttt{data[15..0]} hodnota DAC, \texttt{data[16..31]} ID DAC pro čtení.
    \\\textit{Response}: ACQ.

    %\item[0x05 -- Seq. Readout Start] -- TODO
    
    \item[0x06 -- Acquisition Stop] -- zastaví probíhající akvizici dat (měření).
    \\\textit{Request}: prázdný.
    \\\textit{Response}: ACQ.
    
    \item[0x07 -- HW Command Start] -- spustí interní hardwarový příkaz vyčítacího rozhraní. Přehled HW příkazů je uveden v příloze \ref{chap:app:katherine:hw_commands}.
    \\\textit{Request}: \texttt{data[7..0]} ID HW příkazu.
    \\\textit{Response}: ACQ.

    \item[0x08 -- Sensor Register Setting] -- nastavení registru připojeného \textit{Timepix3} detektoru ve vyčítacím rozhraní. Přehled registrů je uveden v příloze \ref{chap:app:katherine:tpx3_registers}. Pro propsání nastavení registru do detektoru je třeba ještě vykonat interní HW příkaz s ID \textbf{0}.
    \\\textit{Request}: \texttt{data[39..32]} ID registru, \texttt{data[31..0]} hodnota registru.
    \\\textit{Response}: ACQ.
    
    \item[0x09 -- Acquisition Mode Setting] -- nastavení akvizičního módu detektoru.
    \\\textit{Request}: \texttt{data[1..0]} akviziční mód (00 = \textit{ToA \& ToT}, 01 = \textit{pouze ToA}, 10 = \textit{Event \& iTOT}), \texttt{data[7]} povolení \textit{FastToA}\footnote{LSB (bit s nejnižší hodnotou) ToA je \unit{25}{ns}, s aktivovaným FastToA je časové rozlišení zlepšeno na \unit{1.5625}{ns}} (1 = povoleno).
    \\\textit{Response}: ACQ.

    \item[0x0A -- Acquisition Time Setting -- MSB] -- nastaví nejvyšších 32 bitů akvizičního času v desítkách nanosekund.
    \\\textit{Request}: \texttt{data[31..0]} MSB akvizičního času.
    \\\textit{Response}: ACQ.

    \item[0x0B -- Echo Chip ID] -- vrátí ID připojeného detektoru.
    \\\textit{Request}: nic.
    \\\textit{Response}: \texttt{data[31..0]} ID detektoru.

    \item[0x0C -- Get Bias Voltage] -- změří a vrátí bias napětí detektoru.
    \\\textit{Request}: \texttt{data[39..32]} ID biasu.
    \\\textit{Response}: \texttt{data[31..0]} float hodnotu biasu detektoru.

    %\item[0x0D -- Get ADC Voltage] -- 
    %\item[0x0E -- Get Back Read Register] -- 

    \item[0x0F -- Internal DAC Scan] -- naměří a vrátí hodnotu DAC.
    \\\textit{Request}: \texttt{data[7..0]} ID DAC (viz \ref{tab:app:dacs}).
    \\\textit{Response}: \texttt{data[31..0]} float hodnota napětí analogového výstupu převodníku.

    %\item[0x10 -- Set Pixel Config] -- nastaví konfiguraci jednotlivých pixelů detektoru. V příloze \ref{chap:app:katherine:pix_config} je příklad převodu z formátu \texttt{BMC}\footnote{Z angl. \textit{Binary Matrix Configuration}, standardizovaný formát pro konfiguraci pixelů detektorů rodiny Medipix.} do formátu vyžadovaného \textit{Katherine} rozhraním.
    %\\\textit{Request}: nejprve je poslán příkaz z prázdnými daty. Poté \textit{Katherine} očekává 65536 bytů s konfigurací pixelů.
    %\\\textit{Response}: ACQ.

    %\item[0x11 -- Get Pixel Config ] -- příkaz vyčte a vrátí konfiguraci pixelů detektoru.
    %\\\textit{Request}: nic.
    %\\\textit{Response}: posloupnost 65536 bytů s konfigurací pixelů.

    \item[0x12 -- Set All Pixel Config] -- nastaví konfiguraci jednotlivých pixelů detektoru. V příloze \ref{chap:app:katherine:pix_config} je příklad převodu z formátu \texttt{BMC}\footnote{Z angl. \textit{Binary Matrix Configuration}, standardizovaný formát pro konfiguraci pixelů detektorů rodiny Medipix.} do formátu vyžadovaného \textit{Katherine} rozhraním.
    \\\textit{Request}: nejprve je poslán příkaz z prázdnými daty. Poté \textit{Katherine} očekává 65536 bytů s konfigurací pixelů.
    \\\textit{Response}: ACQ.

    \item[0x13 -- Number of Frames Setting] -- nastaví počet snímků, které se naměří, je-li akvizice dat spuštěna v \textit{frame-based} módu (viz \ref{chap:detectors:operation_modes}).
    \textit{Request}: \texttt{data[31..0]} počet snímků.
    \textit{Response}: ACQ.

    \item[0x14 -- Get All DAC Scan] -- vrátí sekvenci 22 naměřených hodnot DAC, seřazených dle čtecího ID DAC (viz \ref{chap:app:katherine:dacs}).
    \\\textit{Request}: nic.
    \\\textit{Response}: 22 response datagramů, kde \texttt{data[31..0]} je float hodnota napětí analogového výstupu převodníku.

    \item[0x15 -- Get HW/Readout Temperature] -- naměří a vrátí teplotu vyčítacího rozhraní.
    \\\textit{Request}: nic.
    \\\textit{Response}: \texttt{data[31..0]} teplota (v °C, float).

    %\item[0x16 -- LED settings] -- 
    
    \item[0x17 -- Get Readout Status] -- vrátí informace o své HW a FW konfiguraci.
    \\\textit{Request}: nic.
    \\\textit{Response}: \texttt{data[7..0]} typ hardware ($0x1$ pro \textit{Katherine} ethernet readout),\\\texttt{data[15..8]} verze HW revize, \texttt{data[31..16]} sériové číslo HW, \texttt{data[47..32]} verze FW.

    \item[0x18 -- Get Communication Status] -- vrátí stav komunikace mezi vyčítacím rozhraním a detektorem.
    \\\textit{Request}: nic.
    \\\textit{Response}: \texttt{data[7..0]} maska komunikačních kanálů, \texttt{data[15..8]} aktuální maximální datový tok mezi vyčítacím rozhraním a detektorem (float, po vynásobení této hodnoty 5 je získána výsledná hodnota v Mb/s), \texttt{data[16]} flag připojení detektoru (1 = přípojen).

    \item[0x19 -- Get Sensor Temperature] -- naměří a vrátí teplotu senzoru připojeného detektoru.
    \\\textit{Request}: nic.
    \\\textit{Response}: \texttt{data[31..0]} teplota (v °C, float).

    \item[0x20 -- Digital Test] -- provede test digitální části detektoru a vrátí výsledek. Je-li výsledek roven 64, pak test proběhl v pořádku.
    \\\textit{Request}: nic.
    \\\textit{Response}: \texttt{data[7..0]} výsledek testu.

    \item[0x28 -- ToA Calibration Setup] -- zapnutí/vypnutí \textit{ToA offset korekce} (viz \ref{chap:detectors:calibration:toa_correction}).
    \\\textit{Request}: \texttt{data[0]} zapnutí \textit{ToA offset korekce} (1 = zapnuto).
    \\\textit{Response}: ACQ.

\end{description}

%********************************************************************************
% Komunikační protokol -- Struktura měřených dat
%********************************************************************************
\subsection{Struktura měřených dat}\label{chap:katherine:protocol:measure_data_structure}
Přenos měřených dat je realizován proudem datagramů ze serveru (vyčítací rozhraní) do klienta (např. handler). Data jsou posílána na předem definovaný port, uložený ve statické konfiguraci \textit{Katherine}\footnote{V době psaní této práce probíhal vývoj podpory pro dynamické nastavování klientského portu (přidáním nového příkazu do řídící části komunikačního protokolu).}. Každý datagram má délku 6 bytů a jeho struktura je znázorněna na obr. \ref{fig:katherine:protocol:data_packet}. Přehled typů datagramů, vč. jejich významu a obsažených dat, je uveden v tabulce \ref{tab:katherine:protocol:data_packet_header}.

\begin{figure}[h]
	\begin{center}
        \begin{bytefield}[endianness=big,bitwidth=0.7em]{48}
            \bitheader{47,0} \\
            \bitbox{12}{\textbf{Hlavička} 47..44} & \bitbox{36}{\textbf{Data} 43..0}
        \end{bytefield}
	\end{center}
	\caption{Struktura datagramu s měřenými daty.}
	\label{fig:katherine:protocol:data_packet}
\end{figure}

\begin{table}[h]
	\begin{center}
		\begin{tabular}{|c|l|l|}
			\hline
            \textbf{Hlavička} & \textbf{Název} & \textbf{Data} \\
			\hline
            0x7 & Nový frame vytvořen & 0x0 \\
            0x4 & Měřená data & Dle konfigurace \\
            0x5 & ToA timestamp offset & ToA offset \\
            & (pouze pro data-driven mód) & \\
            0xC & Aktuální snímek vytvořen & Počet poslaných událostí \\
            0x8 & Frame start timestamp LSB & \unit{32}{b} LSB \\
            0x9 & Frame start timestamp MSB & \unit{16}{b} MSB \\
            0xA & Frame end timestamp LSB & \unit{32}{b} LSB \\
            0xB & Frame end timestamp MSB & \unit{16}{b} MSB \\
            0xD & Počet ztracených pixelů & \unit{44}{b} \\
            0xE & Notifikace o přerušeném měření & 0x0 \\
			\hline
		\end{tabular}
	\end{center}
    \caption{Přehled typů datagramů pro měřená data.}
    \label{tab:katherine:protocol:data_packet_header}
\end{table}

%Výsledný čas ToA (viz \ref{chap:detectors:operation_modes}) je vypočten 
Z pohledu pořadí toku jednotlivých datagramů rozlišujeme dva případy, dle nastaveného vyčítacího módu:
\begin{description}
   \item[Frame-based] -- vyčítání po snímcích (viz \ref{chap:detectors:readout})
   \begin{enumerate}
       \item Klient pošle \textit{Acquisition Start} (0x03) příkaz.
       \item Vyčítací rozhraní zahájí akvizici snímku a pošle datagram o vytvoření nového frame (hlavička 0x7).
       \item Akvizice dat probíhá po dobu zvoleného akvizičního času. Jsou-li detekovány nějaké události, pak pro každou událost je poslán datagram s měřenými daty (hlavička 0x4). V případě zaznamenání příkazu pro přerušení měření, dojde k poslání datagramu s hlavičkou 0xE a měření je ukončeno.
       \item Po ukončení akvizice aktuálního snímku (uzavření \textit{shutteru}) readout pošle timestamp začátku a konce akvizice dat (datagramy s hlavičkami 0x8, 0x9, 0xA a 0xB).
       \item Je poslán datagram s informací o počtu ztracených pixelů (hlavička 0xD) a ukončení aktuálního snímku (hlavička 0xC).
       \item Je-li počet aktuálně pořízených snímků nižší, než celkový počet snímků (nastavený řídícím příkazem 0x13), pak měření pokračuje bodem 2, v opačném případě měření končí.
   \end{enumerate}
   \item[Data-driven] -- režim kontinuální vyčítání dat (viz \ref{chap:detectors:readout})
   \begin{enumerate}
    \item Klient pošle \textit{Acquisition Start} (0x03) příkaz.
    \item Vyčítací rozhraní zahájí akvizici dat (resp. jednoho snímku) a pošle datagram o vytvoření nového frame (hlavička 0x7).
    \item Akvizice dat probíhá po dobu zvoleného akvizičního času. Jsou-li detekovány nějaké události, pak pro každou událost je poslán datagram s měřenými daty (hlavička 0x4). V případě zaznamenání příkazu pro přerušení měření, dojde k poslání datagramu s hlavičkou 0xE a měření je ukončeno.
    \item Po ukončení akvizice aktuálního snímku (uzavření \textit{shutteru}) readout pošle timestamp začátku a konce akvizice dat (datagramy s hlavičkami 0x8, 0x9, 0xA a 0xB).
    \item Je poslán datagram s informací o počtu ztracených pixelů (hlavička 0xD) a ukončení aktuálního snímku (hlavička 0xC).
    \item Měření je ukončeno.
\end{enumerate}
\end{description}

Tabulka \ref{tab:katherine:protocol:measurement_data_structure} znázorňuje formát datagramu s měřenými daty, dle použitého nastavení vyčítacího rozhraní.

\begin{table}[h!]
    \begin{center}
        \includegraphics[width=14cm]{figures/katherine_pixel_measurement_data.png}
        \caption{Struktura měřených dat \cite{katherine_docs}.}
        \label{tab:katherine:protocol:measurement_data_structure}
    \end{center}
\end{table}

%********************************************************************************
% Komunikační modul
%********************************************************************************
\section{Komunikační modul}\label{chap:katherine:comm}
Pro účely řízení komunikačního rozhraní \textit{Katherine} a vyčítání měřených dat byl implementován komunikační modul, který bude popsán v této podkapitole. Aby bylo možné integrovat modul do systému, musí obsahovat implementaci komunikačního interface (viz \ref{chap:handler:detector_layer:commIntf}), které musí být v manifestu zkompilovaného \texttt{jar} archívu explicitně uvedeno (viz \ref{chap:handler:detector_layer:module_init}).

Na obrázku \ref{fig:katherine:comm:arch} je přehled vrstev softwarové architektury komunikačního modulu, které budou dále popsány v následujících podkapitolách.

\begin{figure}[h]
	\begin{center}
		\includegraphics[width=14cm]{figures/katherine_comm_arch.pdf}
		\caption{Softwarová architektura komunikačního modulu, implementujícího komunikační interface.}
		\label{fig:katherine:comm:arch}
	\end{center}
\end{figure}

\subsection{Implementace komunikačního interface}
Tato vrstva slouží jako tzv. fasáda\footnote{Návrhový vzor fasáda (\textit{Facade Pattern}) -- použití pro poskytování zjednodušeného interface pro komplexní části kódu (analogie s pojmem fasáda, známém z architektury).} Katherine kontroléru. Implementace komunikačního interface implementuje všechny jeho metody, vč. poskytování přehledu a podpory vykonávání \textit{ValueCommands} a \textit{ExecutionCommands} (pro jejich definici viz soubor \texttt{detector\_model.json} na přiloženém CD, viz příloha \ref{chap:app:cd}), navazování spojení s detektorem, nahrávání konfigurace apod.

Po zavedení modulu je vytvořena fronta měřených dat, která je přejatá handlerem.

Připojením detektoru vzniká instance Katherine kontroléru dle předané konfigurace. Při odpojení detektoru je kontrolér notifikován, aby mohl uvolnit svoje prostředky (např. uzavřít síťové spojení) a poté je odstraněn.

\subsection{Katherine kontrolér}
Tato vrstva udržuje spojení s detektorem a implementuje všechny potřebné řídící příkazy detektoru.

Odesílání řídících příkazů je realizováno pomocí komponenty \texttt{Control Datagram Sender} (viz obr. \ref{fig:katherine:comm:arch}), který implementuje asynchronní frontu datagramů. Fronta je asynchronně zpracovávána nezávislým vláknem, které datagramy z fronty odesílá na řídící port \textit{Katherine}.

O příjem dat z řídícího socketu se stará komponenta \texttt{Control Datagram Receiver}, která v separátním vlákně přijímá příchozí data. Každý příchozí datagram (8-bytový vektor) je zpracován a přidán do fronty příchozích řídících dat. Komponenta také implementuje blokující metodu pro přijetí dat, která dle zadaného ID příkazu a maximální doby čekání (tzv. \textit{timeout}) hledá požadovanou odpověď ve frontě dat.

Příklad použití obou komponent zmíněných výše je ve zdrojovém kódu \ref{src:katherine:comm:example_get_bias}. Příklad demonstruje vyčtení biasu z detektoru. Nejprve je pomocí \texttt{Control Datagram Sender} odeslán řídící příkaz s ID 0x0C (řádek 3 až 7) a následně pomocí \texttt{Control Datagram Receiver}, resp. pomocí jeho blokující metody popsané výše, je zachycena odpověď, ze které je vyčtena požadovaná hodnota napětí. Jestli se odpověď nepodaří přijmout ve zvoleném timeoutu, je vygenerována výjimka, která musí být vyššími vrstvami programu ošetřena.

\begin{code}[h]
    \begin{minted}[
      frame=single,
        linenos,
        breaklines
      ]{Kotlin}
@Throws(DetectorException::class)
fun getBias(): Float {
    packetSender.send(
        PacketBuilder.builder()
            .commandID(KatherineValueCommands.BIAS.getterCommandId)
            .build()
    )

    val packet = packetReceiver.pollPacketFromQueue(
        KatherineValueCommands.BIAS.getterCommandId,
        RESPONSE_TIMEOUT
    )
    return packet.floatValue
}
    \end{minted}
    \caption{Příklad implementace řídícího příkazu \textit{Katherine} pro vyčtení biasu pomocí komponent \texttt{Control Datagram Sender} a \texttt{Control Datagram Receiver}.}
    \label{src:katherine:comm:example_get_bias}
\end{code}

Poslední komponentou je \texttt{Measurement Datagram Receiver}, který ve svém separátním vlákně přijímá příchozí datagramy na socketu pro detektorem měřená data, jejichž struktura už byla popsána v \ref{chap:katherine:protocol:measure_data_structure}. Každá příchozí událost je zpracována (identifikace hlavičky apod.) a vložena do fronty měřených dat, která je zpracovávána datovým modulem.

%********************************************************************************
% Datový modul
%********************************************************************************
\section{Datový modul}\label{chap:katherine:data}
Datový modul implementuje \texttt{DataPersistence} interface (viz \ref{chap:handler:detector_layer:dataIntf}). Po jeho zavedení do systému je mu handlerem předána fronta dat, kterou po spuštění začne zpracovávat.

Fronta dat obsahuje objekty, které jsou potomky objektu \texttt{AbstractDataFrame} (který je součástí modelu poskytovaných knihoven). V rámci modulu se společnou \textit{code-base} \textit{Katherine} modulů (viz úvod kapitoly \ref{chap:katherine}) byly implementovány tyto třídy, které jsou potomky třídy \texttt{AbstractDataFrame}:
\begin{description}
    \item[\texttt{DataFrame}] -- třída obsahující jednu událost, přenesenou datovým socketem z \textit{Katherine} rozhraní. Objekt obsahuje dekódovanou hlavičku a vlastní data, dle typu datagramu -- viz tabulka \ref{tab:katherine:protocol:data_packet_header}.
    \item[\texttt{ControlFrame}] -- obsahuje informaci generovanou komunikačním modulem, např. o začátku nebo ukončení akvizice apod.
    \item[\texttt{AcqConfigurationFrame}] -- instance tohoto objektu je poslána komunikačním modulem na začátku akvizice dat a obsahuje informace o stavu a konfiguraci detektoru (např. bias, vyčítací mód, teploty apod.).
\end{description}  

Výstupem datového modulu jsou soubory s jednotlivými měřeními ve formátu \texttt{ASCII}, které jsou ukládány do zvoleného adresáře dle konfigurace. Příklad obsahu takového souboru je uveden v příloze \ref{chap:app:katherine:data_example}. Všechny časové údaje jsou v \texttt{UTC} čase. Formát pojmenovávání souborů je \texttt{název\_detektoru\_yyyy\_MM\_dd-HH\_mm\_ss.txt}.

Rovněž byla implementována varianta datového modulu, který ukládá pořízená data do \textit{MongoDB} databáze (viz \ref{chap:arch:technologie:mongodb}).

%********************************************************************************
% Katherine emulátor
%********************************************************************************
\section{Katherine emulátor}\label{chap:katherine:emulator}
Pro účely vývoje a testování byl vyvinut emulátor, které emuluje funkci \textit{Katherine} vyčítacího rozhraním s připojeným \textit{Timepix3} (viz \ref{chap:detectors:medipix_overview:timepix3}) detektorem na síťové vrstvě. Byla implementována podpora pro všechny řídící příkazy uvedené v \ref{chap:katherine:protocol:control_commands}, které jsou relevantní k akvizici dat (tj. např. akviziční čas, akviziční mód, vyčítací mód, bias, počet snímků apod.) a také příkazy pro měření teplot, vyčítání ID senzoru, či provádění testu digitální částí detektoru. Emulátor rovněž podporuje všechny datagramy pro přenos měřených dat (viz \ref{chap:katherine:protocol:measure_data_structure}).

Emulátor byl vyvinut v jazyce \texttt{Java}, resp. \texttt{Kotlin} (viz \ref{chap:arch:technologie:kotlin}).

\subsection{Softwarová architektura}
Na obrázku \ref{fig:katherine:emulator:arch} jsou znázorněny jednotlivé komponenty emulátoru. Základním prvkem je jeho jádro (na obr. \ref{fig:katherine:emulator:arch} jako \textit{emulator core}), které je vytvořeno po spuštění emulátoru. Společně s jádrem je vytvořen také \textit{Control Datagram Receiver}, který přijímá příchozí datagramy na komunikačním socketu detektoru. Příchozí řídící datagram je dále zpracován dle jeho hlavičky a klientovi je odeslána příslušná odpověď.

\begin{figure}[h]
	\begin{center}
		\includegraphics[width=14cm]{figures/katherine_emulator_arch.pdf}
		\caption{Softwarová architektura \textit{Katherine} emulátoru.}
		\label{fig:katherine:emulator:arch}
	\end{center}
\end{figure}

Z pohledu zpracování je možné příkazy rozdělit do tří kategorií:
\begin{description}
    \item[Nastavení akvizičních parametrů] -- do této skupiny spadají všechny příkazy, které nastavují, nebo vracejí konfigurační parametry detektoru, jako je akviziční čas, bias, akviziční a vyčítací mód apod.
    \item[Řízení akvizice] -- do této kategorie patří příkazy \textit{Acquisition Start} (0x03) a \textit{Acquisition Stop} (0x06). Po spuštění akvizice je vytvořena instance \texttt{Acquisition Controller}, který dle konfigurace začne generovat události a posílat je datovým socketem klientovi.
    
    Příkaz \textit{Acquisition Stop} pak přeruší probíhající akvizici.
    \item[Ostatní] -- do této kategorie spadají ostatní příkazy, které nebyly zahrnuty v předchozích kategoriích (např. pro měření teploty, čtení \textit{chip id}, verze FW apod.).
\end{description}

Některé příkazy nebyly implementovány, protože emulace jejich funkce není triviální a je nad rámec požadované funkcionality (jsou to např. příkazy pro nastavování hodnot DAC, registrů apod.). Odpovědí na tyto příkazy bude vždy ACQ (viz \ref{chap:katherine:protocol:control_commands}).

\subsection{Konfigurace a nasazení}
Pro spuštění emulátoru vyčítacího rozhraní \textit{Katherine} je třeba mít nainstalované \texttt{JRE 8}\footnote{\textit{Java SE Runtime Environment, dostupné z\\\url{https://www.oracle.com/technetwork/java/javase/downloads}}.}, nebo vyšší.
Zkompilovanou Java aplikaci je třeba spustit s argumentem cesty ke konfiguračnímu souboru (pro příklad viz zdrojový kód \ref{src:emulator:config}). Konfigurační soubor obsahuje číslo portu, na které bude emulátor naslouchat pro řídící příkazy (\texttt{portCommands}), číslo portu a adresu pro odesílání měřených dat (\textit{portCommands} a \texttt{hots}) a parametr udávající průměrný počet události, které budou v rámci jedné akvizice vygenerovány.

Aplikaci je tedy možné spustit například takto: \mint{bash}{java -jar emulator.jar config.yaml}

\begin{code}[h]
  \begin{minted}[
  frame=single,
  linenos,
  breaklines
  ]{yaml}
portCommands: 1555
portData: 1556
host: 127.0.0.1
avgNumberOfEventsPerAcq: 10
\end{minted}
\caption{\texttt{YAML} konfigurační soubor emulátoru vyčítacího rozhraní \textit{Katherine}.}
\label{src:emulator:config}
\end{code}
\addbibresource{reference.bib}

\chapter{Master}\label{chap:master}
V této kapitole bude čtenář podrobněji seznámen s masterem -- centrálním řídícím prvkem systému Pixnet. Hlavním úkolem masteru je řízení handlerů (viz \ref{chap:handler}), persistence konfigurace a poskytování REST API pro své řízení. Primárním konzumentem API je webová frontend aplikace, poskytující uživatelské rozhraní pro řízení systémů jeho operátorem, pomocí poskytovaného API může být ale řízení systému integrováno i do systémů třetích stran\footnote{Např. DCS v CERN, jak již bylo popsáno v kapitole \ref{chap:arch:hw}}.

Tato komponenta se skládá ze dvou nezávislých částí -- backendového serveru (\ref{chap:master:backend}) frontendového webového klienta (\ref{chap:master:frontend}).

%********************************************************************************
% Backend
%********************************************************************************
\section{Backend}\label{chap:master:backend}
Backendová aplikace mastera, podobně jako handlera, byla implementována pomocí Spring Boot frameworku (viz \ref{chap:arch:technologie:spring}) a obsahuje embedovaný webový server \texttt{Tomcat}. Aplikace komunikuje s handlery, jejichž prostřednictvím řídí jim přiřazené detektory.

\subsection{Softwarová architektura}\label{chap:master:backend:sw_arch}
Na obrázku \ref{fig:master:arch} je znázorněna softwarová architektura backendové aplikace mastera, rozdělená do několika vrstev, které budou v následujících podkapitolách blíže vysvětleny.

\begin{figure}[h]
	\begin{center}
		\vspace*{0.4cm}
		\includegraphics[width=14cm]{figures/master_arch.pdf}
		\caption{Pixnet -- master: softwarová architektura s vrstvami pro (i) persistenci konfiguraci s stavu systému (\textit{Data Repositories}), (ii) poskytování API (\textit{API controllers}),(iii) komponentu pro komunikaci s handlery (\textit{Handlers Communication Component}) a (iv) službu pro pravidelnou aktualizaci stavu handlerů (\textit{Handler Heartbeat Service}).}
		\label{fig:master:arch}
	\end{center}
\end{figure}

\subsubsection{Vrstva pro persistenci konfigurace a stavu systému}
Tato vrstva je zodpovědná za komunikaci s \textit{PostgreSQL} databází (viz \ref{chap:arch:technologie:postgresql}) pomocí \textit{PostgreSQL} \texttt{JDBC}\footnote{Z angl. \textit{Java Database Connectivity} -- API pro programovací jazyk Java, definující přístup klienta k databázi.} ovladače. Pro mapování jednotlivých entit (resp. Java objektů, tzv. \texttt{ORM}\footnote{Z angl. \textit{Object-relational mapping} -- objektově-relační mapování.}) byl implementován framework \textit{Hibernate}.

Vrstva poskytuje repozitáře (viz obr. \ref{fig:master:arch}) pro dotazy nad uloženými daty a pro jejich modifikaci. Rovněž je zodpovědná za automatické vytváření a aktualizaci databázového schématu, připojené databáze. Návrh schématu bude popsán v \ref{chap:master:backend:db}.

V databázi jsou společně se záznamem detektoru uloženy také jeho moduly (resp. implementace komunikačního a datového interface).

\subsubsection{Komponenta pro komunikaci s handlery}
Komponenta poskytuje metody pro komunikaci s handlery pomocí jejich API, popsaném v \ref{tab:handler:api_detectors}. Umožňuje přiřazování detektorů jednotlivým handlerům, navazování spojení s detektory, zjišťování jejich stavu a funkcionality (vč. seznamů podporovaných příkazů), vykonávání jednotlivých příkazů, nahrávání souborů apod.

\subsubsection{Vrstva s API kontroléry}\label{chap:master:backend:sw_arch:api}
Tato vrstva se sestává z REST API kontrolérů pro poskytování API metod pro řízení mastera. Těmito kontroléry jsou:
\begin{description}
    \item[\texttt{Subnetworks Controller}] -- poskytuje CRUD\footnote{Z angl. \textit{Create Read Update and Delete}, operace pro vytváření, čtení, aktualizaci a odstranění.} operace nad entitou podsítí (viz \ref{chap:arch:hw}). Pro přehled endpointů viz tabulka \ref{tab:master:api_subnetworks}.

    \begin{table}[h!]
        \begin{center}
            \begin{tabular}{|c|c|l|}
                \hline
          \textbf{HTTP} & & \\
          \textbf{Metoda} & \textbf{Endpoint} & \textbf{Popis} \\
            \hline
            POST & \texttt{/add} & Přidá novou podsíť \\
            GET & \texttt{/getAll} & Vrátí seznam všech podsítí \\
            DELETE & \texttt{/deleteAll} & Odstraní všechny podsítě \\
            DELETE & \texttt{/deleteById} & Odstraní podsíť podle zadaného ID \\
            \hline
            \end{tabular}
        \end{center}
        \caption{Endpointy komponenty \textit{Subnetworks Controller} (všechny endpointy mají prefix \texttt{/api/subnetwork}).}
        \label{tab:master:api_subnetworks}
    \end{table}

    \item[\texttt{Handlers Controller}] -- poskytuje CRUD operace nad entitou handlerů. Pro přehled endpointů viz tabulka \ref{tab:master:api_handlers}.

    \begin{table}[h!]
        \begin{center}
            \begin{tabular}{|c|c|l|}
                \hline
          \textbf{HTTP} & & \\
          \textbf{Metoda} & \textbf{Endpoint} & \textbf{Popis} \\
            \hline
            POST & \texttt{/add} & Přidá nový handler \\
            GET & \texttt{/getAll} & Vrátí seznam všech handlerů \\
            DELETE & \texttt{/deleteAll} & Odstraní všechny handlery \\
            DELETE & (pouze prefix) & Odstraní handler podle zadaného ID \\
            \hline
            \end{tabular}
        \end{center}
        \caption{Endpointy komponenty \textit{Handlers Controller} (všechny endpointy mají prefix \texttt{/api/handler}).}
        \label{tab:master:api_handlers}
    \end{table}

    \item[\texttt{Detectors Controller}] -- kromě poskytování CRUD operací nad entitou detektorů také poskytuje metody pro přiřazování detektorů handlerům, navazování spojení s detektory, vykonávání jejich příkazů (viz \textit{ValueCommands} a \textit{ExecutionCommands}, \ref{chap:handler:detector_layer:commIntf}). Pro přehled endpointů viz tabulka \ref{tab:master:api_detectors}.

    \begin{table}[h!]
        \begin{center}
            \begin{tabular}{|c|c|l|}
                \hline
          \textbf{HTTP} & & \\
          \textbf{Metoda} & \textbf{Endpoint} & \textbf{Popis} \\
            \hline
            POST & \texttt{/add} & Přidá nový detektor \\
            GET & \texttt{/getAll} & Vrátí seznam všech detektorů \\
            GET & \texttt{/getById} & Vrátí jeden detektor \\
            GET & \texttt{/getDetailById} & Vrátí detail jednoho detektoru \\
            DELETE & (pouze prefix) & Odstraní detektor podle zadaného ID \\
            POST & \texttt{/bindToHandler} & Přiřadí detektor handleru \\
            POST & \texttt{/unbindFromHandler} & Odebere detektor handleru \\
            POST & \texttt{/connect} & Naváže spojení mezi handlerem a detektorem \\
            POST & \texttt{/disconnect} & Ukončí spojení mezi handlerem a detektorem \\
            POST & \texttt{/executeValueCommand} & Vykoná \texttt{ValueCommand} detektoru \\
            POST & \texttt{/executeExecutionCommand} & Vykoná \texttt{ExecutionCommand} detektoru \\
            POST & \texttt{/uploadFile} & Nahraje soubor do detektoru \\
            \hline
            \end{tabular}
        \end{center}
        \caption{Endpointy komponenty \textit{Detectors Controller} (všechny endpointy mají prefix \texttt{/api/detector}).}
        \label{tab:master:api_detectors}
    \end{table}    

    \item[\texttt{Swagger API DOC Controller}] -- kontrolér pro poskytování webové API dokumentace, dostupné z endpointu \texttt{/apiDoc}. Tato komponenta byla implementována, podobně jako u handleru (viz \ref{chap:handler:spring:swagger}), za pomocí \textit{Springfox} \cite{springfox} knihovny. Kromě přehledu všech endpointů webové rozhraní také poskytuje definici datového modelu API (ve formátu JSON) a umožňuje manuální volání jednotlivých endpointů. Endpoint pro stažení Open\-API specifikace ve formátu JSON je \texttt{/v2/api-docs}.
\end{description}

\subsubsection{Handlers Heartbeat Service}
Tato služba slouží k periodickému zjišťování stavu jednotlivých handlerů (a tranzitivně i detektorů) systému. V pravidelných intervalech (defaultně nastaveno na \unit{10}{s}) je navázáno spojení se všemi handlery pomocí jejich \texttt{/status} endpointu (viz \ref{chap:handler:spring:status_api}) a jsou aktualizována stavová data v databázi.

\subsection{Návrh databáze}\label{chap:master:backend:db}
Na obrázku \ref{fig:master:db_schema} je znázorněno schéma relačního modelu \textit{PostgreSQL} databáze, kde jsou znázorněny jednotlivé entity, vč. jejich atributů, a relace mezi nimi pomocí cizích klíčů.

Ve schématu jsou dále definována různá integritní omezení, která vynucují strukturu dat a pravidla mezi relacemi entit -- např. že handler musí mít přiřazení právě jednu podsíť, že detektor může být přiřazen handleru ze stejné podsítě apod.

\begin{figure}[h]
	\begin{center}
		\vspace*{0.4cm}
		\includegraphics[width=15cm]{figures/master_db.png}
		\caption{Pixnet -- master: databázové schéma.}
		\label{fig:master:db_schema}
	\end{center}
\end{figure}

\subsection{Konfigurace a nasazení}
Pro spuštění mastera je třeba mít nainstalované \texttt{JRE 8}\footnote{\textit{Java SE Runtime Environment, dostupné z\\\url{https://www.oracle.com/technetwork/java/javase/downloads}}.} nebo vyšší.
Zkompilovanou java aplikaci je třeba spustit s argumentem \texttt{masterConfig} obsahující cestu ke konfiguračnímu souboru (pro příklad viz zdrojový kód \ref{src:master:config}), obsahujícího port na kterém bude aplikace naslouchat příchozí spojení.

Aplikaci je tedy možné spustit například takto: \mint{bash}{java -jar master.jar masterConfig=config.yaml}

\begin{code}[h]
  \begin{minted}[
  frame=single,
  linenos,
  breaklines
  ]{yaml}
# Connection parameters
portToListen: 8080
\end{minted}
\caption{\texttt{YAML} konfigurační soubor mastera.}
\label{src:master:config}
\end{code}

%********************************************************************************
% Frontend
%********************************************************************************
\section{Frontend}\label{chap:master:frontend}
Pomocí frameworku \textit{ReactJS} (viz \ref{chap:arch:technologie:react}) byl implementován webový klient, který implementuje API, poskytované backendovou aplikací mastera (viz \ref{chap:master:backend:sw_arch:api}). 

Jedná se o tzt. \textit{single-page} aplikaci, tj. aplikaci která načte jednu stránku, kterou pak dynamicky aktualizuje dle interakcí uživatele (ev. dalších událostí) bez nutnost stahování nové stránky ze serveru. Jak již bylo vysvětleno v kapitole \ref{chap:arch:technologie:react}, každá \textit{ReactJS} aplikace je založena na komponentách. Každá komponenta má svoje vlastnosti a svůj stav. Komponenty lze opětovně používat a je možné je libovolně zapouzdřovat. Při aktualizaci stavu některé z komponent, \textit{ReactJS} framework změnu detekuje a provede překreslení afektovaných částí uživatelského rozhraní.

Pro navigaci po stránce aplikace používá knihovnu \textit{react-router}\footnoteUrl{https://reacttraining.com/react-router/web}, která mění obsah stránky renderováním těchto komponent:

\begin{description}
    \item[Subnetworks] -- komponenta vizualizující seznam podsítí a umožňuje přidávání nových podsítí a odebírání existujících.
    \item[Handlers] -- tato komponenta zobrazuje seznam přidaných handlerů, včetně podseznamu jim přiřazených detektorů, stavu apod. Komponenta rovněž umožňuje přidání nových hadlerů a odstranění existujících. Pro screenshot viz obr. \ref{fig:master:frontend:handlers}.
    \item[Detectors] -- komponenta zobrazující seznam všech detektorů (viz \ref{fig:master:frontend:detectors}). Dále umožňuje přidávání nových detektorů a odebírání existujících.
    \item[Detector detail] -- komponenta s detailem detektoru (viz obr. \ref{fig:master:frontend:detector_detail}). Komponenta uživateli nabízí všechny operace nad detektorem, nabízené API backendové aplikace mastera (tj. např. přiřazování detektoru dostupnému handleru, navazování spojení s detektorem, vykonávání jeho příkazů apod.).
\end{description}

\begin{figure}[h]
	\begin{center}
        \includegraphics[width=15cm]{figures/master_handlers.png}
	\end{center}
	\caption{Master -- frontend: screenshot seznamu handlerů, dostupného z endpointu \texttt{/handlers}.}
	\label{fig:master:frontend:handlers}
\end{figure}

\begin{figure}[h]
	\begin{center}
        \includegraphics[width=10cm]{figures/master_detectors.png}
	\end{center}
	\caption{Master -- frontend: screenshot seznamu detektorů, dostupného z endpointu \texttt{/detectors}.}
	\label{fig:master:frontend:detectors}
\end{figure}

\begin{figure}[h]
	\begin{center}
        \includegraphics[width=15cm]{figures/master_detector_detail.png}
	\end{center}
    \caption{Master -- frontend: screenshot detailu detektoru, dostupného z endpointu \texttt{detectors/\{ID detektoru\}}.}
	\label{fig:master:frontend:detector_detail}
\end{figure}

\subsection{Nasazení}
Nasazení aplikace je jednoduché -- stačí jenom její build (k dispozici na přiloženém CD) nahrát na webový server. Build obsahuje \texttt{index.html} společně s vlastní aplikací v \texttt{bundle.js} a dalšími soubory (obrázky, CSS styly apod.).

Aplikace byla vyvinuta pomocí systému na správu závislostí \texttt{Yarn}\footnoteUrl{https://yarnpkg.com}. Pro vývojové účely je možné aplikaci spustit příkazem \texttt{yarn start}, nebo příkazem \texttt{yarn build} je možné udělat build aplikace, nasaditelný na webový server.
\addbibresource{reference.bib}

\chapter{Testování}\label{chap:test}
\todo
\addbibresource{reference.bib}

\chapter{Závěr}\label{chap:zaver}
\todo

\bibliographystyle{csplainnat}

{
%JZ: 11.12.2008 Kdo chce mit v~techto ukazkovych odkazech take odkaz na CSTeX:
\def\CS{$\cal C\kern-0.1667em\lower.5ex\hbox{$\cal S$}\kern-0.075em $}
\bibliography{reference}
}


%%%%%%%%%%%%%%%%%%%%%%%%%% 
% Přílohy
\appendix	
%\chapter{Závislosti modulu Handler}\label{chap:app:handler_dependencies}
\chapter{Přílohy vyčítacího rozhraní Katherine}\label{chap:app:katherine}

\section{Přehled DAC vyčítacího rozhraní Katherine}\label{chap:app:katherine:dacs}
\begin{table}[h]
	\begin{center}
		\begin{tabular}{|c|c|c|}
			\hline
            \textbf{Název} & \textbf{ID pro čtení} & \textbf{ID pro zápis} \\
			\hline
            Ibias\_Preamp\_ON & 1 & 0 \\
            Ibias\_Preamp\_OFF & 2 & 1 \\
            VPreamp\_NCAS & 3 & 2 \\
            Ibias\_Ikrum & 4 & 3 \\
            Vfbk & 5 & 4 \\
            Vthreshold\_fine & 6 & 5 \\
            Vthreshold\_coarse & 7 & 6 \\
            Ibias\_DiscS1\_ON & 8 & 7 \\
            Ibias\_DiscS1\_OFF & 9 & 8 \\
            Ibias\_DiscS2\_ON & 10 & 9 \\
            Ibias\_DiscS2\_OFF & 11 & 10 \\
            Ibias\_PixelDAC & 12 & 11 \\
            Ibias\_TPbufferIn & 13 & 12 \\
            Ibias\_TPbufferOut & 14 & 13 \\
            VTP\_coarse & 15 & 14 \\
            VTP\_fine & 16 & 15 \\
            Ibias\_CP\_PLL & 17 & 16 \\
            PLL\_Vcntrl & 18 & 17 \\
            BandGap output & 28 & - \\
            BandGap\_Temp & 29 & - \\
            Ibias\_dac & 30 & - \\
            Ibias\_dac\_cas & 31 & - \\
			\hline
		\end{tabular}
	\end{center}
	\caption{Přehled DAC (digitálně analogových převodníků) vyčítacího rozhraní Katherine.}
	\label{tab:app:dacs}
\end{table}

\section{Přehled HW příkazů vyčítacího rozhraní Katherine}\label{chap:app:katherine:hw_commands}
\begin{table}[h]
	\begin{center}
		\begin{tabular}{|c|l|}
			\hline
            \textbf{ID} & \textbf{Název} \\
			\hline
            0 & Sensor Config Registers Update \\
            1 & Internal DAC Update \\
            2 & Internal DAC Back Read \\
            3 & Timer Read \\
            4 & Timer Set \\
            5 & Reset Matrix Sequential \\
            6 & Stop Matrix Command \\
            7 & Load Column Test Pulse Register \\
            8 & Read Column Test Pulse Register \\
            9 & Load Pixel Register Configuration \\
            10 & Read Pixel Register Configuration \\
            11 & Read Pixel Matrix Sequential Setting \\
            12 & Read Pixel Matrix Data-Driven Setting \\
            13 & Chip ID Read \\
            14 & Output Block Config Update \\
            15 & Digital Test \\
			\hline
		\end{tabular}
	\end{center}
	\caption{Přehled HW příkazů vyčítacího rozhraní Katherine.}
	\label{tab:app:hw_commands}
\end{table}

\section{Přehled registrů detektoru Timepix3 v rámci vyčítacího rozhraní Katherine}\label{chap:app:katherine:tpx3_registers}
\begin{table}[h!]
	\begin{center}
		\begin{tabular}{|c|l|}
			\hline
            \textbf{ID} & \textbf{Název} \\
			\hline
            0 & Test Pulse Period \\
            1 & Number Test Pulses \\
            2 & Out Block Config \\
            3 & PLL Config \\
            4 & General Config \\
            5 & SLVS Config \\
            6 & Power Pulsing Pattern \\
            7 & SetTimer 15..0 \\
            8 & SetTimer 31..16 \\
            9 & SetTimer 47..32 \\
            10 & Sense DAC Selector \\
            11 & Ext DAC Selector \\
			\hline
		\end{tabular}
	\end{center}
	\caption{Přehled registrů detektoru Timepix3 v rámci vyčítacího rozhraní Katherine.}
\end{table}

\section{Konfigurace pixelů detektoru}\label{chap:app:katherine:pix_config}
\begin{code}[h!]
\begin{minted}[
frame=single,
linenos,
breaklines
]{Kotlin}
fun bmcToMatrixConfig(bmc: ByteArray): ByteArray {
    assert(bmc.size == 65536)
    val buff = IntArray(16384)

    for (i in bmc.indices) {
        var y = i / 256
        val x = i % 256
        val tmp = bmc[i]
        y = 255 - y
        buff[64 * x + (y shr 2)] = buff[64 * x + (y shr 2)] or (tmp.toInt() shl 8 * (3 - y % 4))
    }

    val out = ByteArray(65536)
    var k = 0
    for (mem in buff) {
        val b = ByteBuffer.allocate(4).putInt(mem).array()
        for (i in 0..3) {
            out[k + 3 - i] = b[i]
        }
        k += 4
    }

    return out
}
\end{minted}
\caption{Ukázka kódu pro převod pixelové konfigurace detektoru z formátu \texttt{BMC} do formátu podporovaného vyčítacím rozhraním \textit{Katherine}.}
\end{code}

\section{Příklad výstupních dat datového modulu.}\label{chap:app:katherine:data_example}
\begin{code}[h!]
\begin{minted}[
frame=single,
linenos,
breaklines
]{Bash}
# Start of measurement: 2018.12.11 08:51:18.363+0000
# Acq start timestamp: 1544518278053
# Chip ID: M7-W0005
# Acq mode: ToA & ToT with FastToA
# Fast VCO enabled: true
# Readout mode: Data driven
# Bias: 230,000000
# Temperature readout: 52,125000
# Temperature sensor: 82,522804
# Payload older: PIX_ID	ToA	ToT
# NEW_FRAME_ESTABLISHED
21541	54657847	1
22060	54657847	3
22310	54657840	4
36548	9197790130	8
36810	9197790132	75
35266	9197790130	14
36549	9197790129	16
36553	9197790125	67
35522	9197790132	23
36809	9197790130	8
35778	9197790135	8
...
3973	9674803477	9
4239	9674803471	10
3983	9674803465	24
4226	9674803479	19
5262	9674803485	27
4495	9674803477	6
4751	9674803474	23
5007	9674803479	22
29064	9846817974	5
29319	9846817971	10
29320	9846817966	33
# Start of frame timestamp: 2
# End of frame timestamp: 400000002
# Lost pixels: 0
# Frame 0 finished
# Sent pixels: 859

\end{minted}
\caption{Příklad výstupních dat datového modulu.}
\end{code}

\printnomenclature
\label{apx:zkratky}

\chapter{Obsah přiloženého CD}\label{chap:app:cd}

\definecolor{fblue}{RGB}{92,144,192}
\definecolor{fgreen}{RGB}{34,162,70}

\def\Size{4pt}
\newcommand\myfolder[2][fblue]{%
\begin{tikzpicture}[overlay]
%\begin{scope}[xshift=20pt]
%\tikzset{
\filldraw[draw=folderborder,top color=folderbg!50,bottom color=folderbg]
      (-1.05*\Size,0.2\Size+5pt) rectangle ++(.75*\Size,-0.2\Size-5pt);  
    \filldraw[draw=folderborder,top color=folderbg!50,bottom color=folderbg]
      (-1.15*\Size,-\Size) rectangle (1.15*\Size,\Size);
%\end{scope}  
\end{tikzpicture}%
\makebox[2cm]{\raisebox{-3pt}{{\ttfamily#2}}}%
%}
}


\begin{figure}[th!]
\begin{center}
\begin{forest}
  for tree={
    font=\ttfamily,
    grow'=0,
    child anchor=west,
    parent anchor=south,
    anchor=west,
    calign=first,
    inner xsep=7pt,
    edge path={
      \noexpand\path [draw, \forestoption{edge}]
      (!u.south west) +(7.5pt,0) |- (.child anchor) \forestoption{edge label};
    },
    before typesetting nodes={
      if n=1
        {insert before={[,phantom]}}
        {}
    },
    fit=band,
    before computing xy={l=15pt},
  }  
[CD/
	[pixnet/
		[commons/ - modul se společnými závislosti]
		[detector\_communication\_intf/ - modul s komunikačním interface]
		[detector\_communication\_katherine/ - komunikační modul Katherine]
		[detector\_data\_persistence\_intf/ - modul s datovým interface]
		[handler/ - modul s handlerem]
		[katherine\_commons/ - společné závislosti Katherine modulů]
		[katherine\_emulator/ - emulátor Katherine]
		[katherine\_persistence\_file/ - datový modul Katherine]
		[master/ - modul s backendovou aplikací mastera]
	]
	[pixnet\_frontend/ - webová single-page aplikace]
	[text/
		[src/ - adresář se zdrojovými soubory tohoto dokumentu]
		[master-thesis-jakub-begera-2019.pdf - tato práce ve formátu PDF]
		[abstract\_cz.txt - abstrakt česky]
		[abstract\_en.txt - abstrakt anglicky]
	]
]
\end{forest}
\end{center}
\caption{Obsah přiloženého CD}
\label{fig:attached-cd}
\end{figure}

\end{document}
