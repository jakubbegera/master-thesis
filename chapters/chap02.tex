\addbibresource{reference.bib}

\chapter{Úvod do hybridních částicových pixelových detektorů}\label{chap:detectors}
Ionizující záření je lidskými smysli nedetekovatelné, avšak jeho studie nám umožňuje pochopit podstatu hmoty, její vlastnosti a interakce. To lidstvu umožnilo mnohé aplikace, jako je na příklad protonová terapie \cite{tpx_app_radiotherapy}, defektoskopie nebo zkoumání pravosti uměleckých děl. První pokusy o detekci ionizujícího záření sahají do počátku 20. století, kde pomocí mlžné komory se prvně podařilo zachytit trajektorii nabitých částic. Rozvoj polovodičové technologie dal vzniku novým detekčním technologiím až po v současné době nejpokrokovějším - pixelovým detektorům.

Existuje celá řada částicových pixelových detektorů, ale v této kapitole budou popsány jen hybridní pixelové detektory, pro které je typické, že se skládají ze dvou nezávisle vyrobených částí - senzoru a vyčítacího čipu. To oproti monolitickým detektorů, kde vyčítací elektronika je součástí senzoru přináší řadu výhod, jako na příklad snížení výrobních nákladů nebo možnost kombinace vyčítacího čipu se senzory různých materiálů (\textit{Si}, \textit{GaAs}, \textit{CaTe} apod.) a tlouštěk (vetšinou $300\mu m$, nebo $500\mu m$).

Na tomto místě je třeba zmínit, že existuje více druhů těchto detektorů (\textit{AGH Fermilab, Pilatus, Philips Chromaix} apod.)\cite{detectors_review}, v této práci budou použity použity pouze detektory z rodiny detektorů Medipix.

%********************************************************************************
% Hardwarová architektura
%********************************************************************************
\section{Hardwarová architektura}
\begin{figure}[th]
	\begin{center}
		\includegraphics[width=12cm]{figures/det_chip.png}
		\caption{Struktura hybridního polovodičového pixelového detektoru Timepix3, skládající se z vyčítacího čipu a polovodičového senzoru (převzato z~\cite{PlatkevicDisertace}).}
		\label{fig:det:chip}
	\end{center}
\end{figure}
Většina hybridních částicových pixelových detektorů rodiny Medipix obsahuje matici $256\times256$ pixelů. Každý z nich má stanu o délce $55~\mu m$, takže senzor čítající $65536$ má plochu $1.4 \times 1.4 cm^2$. 

Na obrázku \ref{fig:det:chip} je znázorněna struktura detektoru Timepix3. Vrchní část detektoru tvoří polovodičový senzor, který je nejčastěji vyroben z křemíku, ale výjimkou není také \textit{GaAs} nebo \textit{CaTe}. Jednotlivé pixely senzoru jsou spojeny s integrovaným \texttt{ASIC}\footnote{z angl. Application Specific Integrated Circuit} vyčítacím čipem pomocí technologie zvané \textit{Bump-Bounding}. Vyčítací čip je pak propojen se základní deskou pomocí \textit{wire-bound}, z které je ještě přivedeno měřící napětí na senzor detektoru (tzv. \textit{bias}).


%********************************************************************************
% Princip detekce
%********************************************************************************
\section{Princip detekce}
\begin{figure}[th]
	\begin{center}
		\includegraphics[width=10cm]{figures/det_recombination.png}
		\caption{Princip detekce ionizujícího záření detektorem Timepix3 (převzato z~\cite{PlatkevicDisertace}).}
		\label{fig:det:recombination}
	\end{center}
\end{figure}

Princip detekce ionizujícího záření pixelovými detektory je založen na známém jevu detekce ionizujícího záření v polovodiči. 

Jako náhradní schéma jednoho pixelu si lze představit diodu zapojenou v závěrném směru, kterou bez přítomnosti ionizujícího záření protéká minimální proud. Vnikne-li do senzoru ionizující částice a dojde k její interakci se senzorem, resp. část její energie je deponována do polovodičového objemu senzoru, dojde v senzoru ke vzniku elektron-děrových páru a díky lavinovému efektu i k následnému otevření PN přechodu (viz. na obr. \ref{fig:det:recombination}, kde červená šipka znázorňuje interagující částici, elektrony jsou znázorněny žlutě, modře díry).

Vzniklý proudový impulz je měřícím odporem převeden na napětí, které je komparátorem porovnáno s prahovým napětím (tzv. \textit{threshold}). Výsledek této komparace je dále CMOS obvodem zpracován, dle použitého měřícího módu, jak bude ukázáno v kapitole \ref{chap:detectors:operation_modes}.

Na rozdíl od CCD technologii, CMOS readout \textit{Timepix}/\textit{Medipix} detektorů negeneruje temný proud\footnote{Termín charakterizující vyčítací šum u CCD snímačů. Obvykle je udáván v elektronech za sekundu při konstantní teplotě a ve tmě.}, díky odstínění signálu od šumu pomocí komparačního napětí. To znamená, že doba jedné akvizice je teoreticky neomezena, protože detektor je schopný detekovat jen ty částice, jejíchž deponovaná energie (resp. amplituda vzniklého napěťového pulzu) je větší, než \textit{threshold}.

%********************************************************************************
% Operační módy detektoru
%********************************************************************************
\section{Operační módy detektoru}\label{chap:detectors:operation_modes}
\begin{figure}[th]
	\begin{center}
		\includegraphics[width=14cm]{figures/det_pix.png}
		\caption{Zpracování signálu pixelem detektoru dle nastaveného módu (\textit{Medipix}, \textit{ToT} a \textit{ToA}) (převzato z~\cite{PlatkevicDisertace}).}
		\label{fig:det:modes}
	\end{center}
\end{figure}


V této podkapitole bude vysvětlena většina operačních módu, ve kterých detektory rodiny \textit{Medipix} jsou schopny pracovat. 

Jak už bylo popsáno v předchozí kapitole, interagovaná částice vyvolá napěťový impulz, jehož tvar koreluje s deponovanou energií. Pro účely analýzy se ale používá pouze binární informace o překročení prahového napětí v čase. Výsledek této analýzy je po jejím dokončení uložen ve 14-bitovém registru pixelu.

Na obr. \ref{chap:detectors:operation_modes} je znázorněn příklad zpracování analýzy signálu následujícími módy:
\begin{description}
    \item[Medipix mód (Counting mód)] V tomto módu je čítač inkrementován v každém cyklu měřící frekvence, pokud měřící napětí překročilo prahové napětí pixelu. Na konci akvizice pak hodnota čítače odpovídá počtu zaznamenaných částic.
    \item[Time-Over-Threshold (ToT)] Pracuje-li pixel v tomto módu, pak jeho čítač je inkrementování v každém cyklu měřící frekvence, pokud měřící napětí je vyšší, než prahové napětí pixelu. Hodnota uložená v čítači odpovídá deponované energii interagovaných částic. Mezi energií a \texttt{ToT} je nelineární závislost a její zkoumání je předmětem energetické kalibrace detektoru, jak bude ukázáno v kapitole \ref{chap:detectors:calibration:tot}. Tento mód má široké spektrum aplikací, na příklad \cite{tot_app_counting} nebo \cite{tpx_app_radiotherapy}.
    \item[Time-of-Arrival (ToA)] Tímto módem disponují pouze detektory \textit{Timepix} a \textit{Timepix3}, avšak nesdílí stejný princip. Zatímco \textit{Timepix} detektor začne inkrementovat čítač v každém cyklu měřící frekvence po první náběžné hraně z komparátoru, \textit{Timepix3} na náběžnou hranu uloží do 14-bitového registru aktuální časové razítko z hodin detektoru. V obou případech \texttt{ToA} udává čas první interakce částice v dané akvizici.
\end{description}

%********************************************************************************
% Vyčítání naměřených dat
%********************************************************************************
\section{Vyčítání naměřených dat}\label{chap:detectors:readout}
Jednotlivé detektory rodiny \textit{Medipix} mají různou hardwarovou podporu pro vyčítání naměřených dat. Detektory vždy podporují alespoň jeden z těchto módů:
\begin{description}
	\item[Frame-Based] Pracuje-li detektor v tomto módu, pak jsou všechny registry čítačů pixelů vyčítány najednou, po dokončení aktuálního snímku. Vždy je třeba vyčíst všechny pixely bez ohledu na naměřenou hodnotu.
	\item[Data-Driven] Tento mód, také označovaný jako \textit{Event-Driven}, byl prvně použit v detektoru \textit{Timepix3}. Pracuje-li detektor v tomto módu, pak v průběhu akvizice dat (resp. když \textit{shutter} signál na nastaven na úroveň \texttt{HIGH}) každý pixel po zpracování události notifikuje readout interface o tom, že nová data jsou připravena k vyčtení a readout interface je pak bez prodlení vyčte a dále zpracuje.
\end{description}

Na obrázku \ref{fig:det:frame_vs_event_driven} je vidět hlavní motivace pro zavedení podpory \textit{Data-Driven} módu u detektoru \textit{Timepix3}. Ukázalo se, že \textit{Data-Driven} mód je efektivnější při takových měření, kde okupance snímků je menší než zhruba $50\%$. Po překročení této meze je efektivnější použití \textit{Frame-Based} módů, protože není třeba přenášet souřadnice zasažených pixelů. Podle \cite{timepix3} vyčítací čas může být definován následovně:
\begin{equation}\label{eq:det:readout_time}
	T_{readout} = N_{pixels}*bits_{pixel}/BW
\end{equation}
kde:
\begin{changemargin}{1.5cm}{1cm} 
	\begin{itemize}
		\item[$N_{pixels}$] je počet pixelů které je potřeba vyčíst (pro \textit{Frame-Based} mód jsou to všechny pixely detektoru ($256\times256$) a pro \textit{Data-Driven} je to počet zasažených pixelů),
		\item[$bits_{pixel}$] je počet bitů na pixel ($28 b$ v \textit{Frame-Based} módu a $28b + 16b$ v \textit{Data-Driven} módu kvůli nutnosti přenášení adresy pixelu) a
		\item[$BW$] je počet bytů za vteřinu, které je možné vyčíst z detektoru (\textit{bandwidth}). 
	\end{itemize}
\end{changemargin}

\begin{figure}[th]
	\begin{center}
		\includegraphics[width=14cm]{figures/det_frame_vs_event_driven.png}
		\caption{Doba vyčítání detektoru za použití \textit{Frame-based} (non-sparse) a \textit{Data-driven} (sparse) módu (převzato z~\cite{timepix3}).}
		\label{fig:det:frame_vs_event_driven}
	\end{center}
\end{figure}

%********************************************************************************
% Kalibrace
%********************************************************************************
\section{Kalibrace}\label{chap:detectors:calibration}

\subsection{Energetická kalibrace}\label{chap:detectors:calibration:tot}