\addbibresource{reference.bib}

\chapter{Návrh architektury}\label{chap:arch}
V této kapitole bude čtenář seznámen s návrhem a koncepcí softwarového systému \textbf{Pixnet} - software pro distribuované řízení sítě částicových pixelových detektorů, který byl navržen a implementován v rámci této práce. V této kapitole bude popsána motivace pro vznik tohoto systému a budou představeny jednotlivé komponenty systému a jejich vzájemné interakce. Pro detailnější popis návrhu a implementace komponent viz kapitoly \todo (přidat ref na handler až master).

%********************************************************************************
% Motivace
%********************************************************************************
\section{Motivace}
Hlavní motivací pro vznik tohoto systému je fakt, že moderní částicové pixelové detektory jsou schopné generovat vysoký datový tok, na příklad \textit{Timepix3} má teoretické maximum \unit{5,12}{Gb/s} (viz \ref{chap:detectors:medipix_overview:timepix3}), takže nedistribuovaný systém, který by operoval na jedné instanci, by nebyl schopný zpracovat datový tok, který síť o vice detektorech je schopná generovat. 

Zde je možné namítnout, že každý systém je možné škálovat vertikálně\footnote{Škálováním v kontextu počítačových systému rozumíme změnu vlastností daného systému za účelem zvýšení, nebo snížení jeho výpočetního výkonu (ev. jiného sledovaného parametru). Zatímco u vertikálního škálování měníme vlastnosti jednoho uzlu systému (na příklad přidáváním procesorů, pamětí, kapacity úložiště apod.), u horizontálního škálování přidáváme jednotlivé uzly - samostatné  jednotky (na př. počítače). Pro úplnost je třeba doplnit že vertikální škálování má své omezení z hlediska použitého hardware, u horizontálního škálování žádná taková omezení nejsou.}. Zatímco cena škálování horizontálně škálovatelného systému je lineární závislost výpočetního výkonu na ceně, u vertikálně škálovatelného systému tato závislost roste exponenciálně. Jelikož vertikální škálování takového systému je  vysoce neefektivní, nebude dále uvažováno a tato práce se bude věnovat jenom návrhu a implementaci horizontálně škálovatelného řešení.

 Další motivací pro vytvoření tohoto systému je možnost řízení heterogenní sítě detektorů homogenním způsobem. Heterogenní sítí detektorů rozumíme takovou síť, ve které jsou detektory různých typů (na příklad \textit{Timepix}, \textit{Timepix3} apod, viz \ref{chap:detectors:medipix_overview}), komunikující různými komunikačními protokoly prostřednictvím různých vyčítacích rozhraních (na příklad \textit{Katherine}, \textit{ATLAS Pix} apod, viz \ref{chap:detectors:readouts}). V další části textu bude detailně popsána navržená a implementovaná modulová architektura, která výše zmíněné umožňuje.

 Pro potřeby experimentu \textit{ATLAS TPX} byl již vyvinut software \cite{atlastpx_sw,BegeraBcThesis2016} pro řízení sítě detektorů \textit{Timepix} \ref{chap:detectors:medipix_overview:timepix}, prostřednictvím vyčítacího rozhraní \textit{ATLAS Pix} \ref{chap:detectors:readouts:atlaspix}. Software však nevyhovuje požadavkům zmíněných výše:
 \begin{enumerate}[label=(\roman*)]
     \item \textbf{Škálovatelnost} - systém je navržen bez možnosti horizontálního škálování. Všechny detektory sítě jsou řízeny z jednoho uzlu a všechna vygenerovaná data jsou jím zpracovávány. Možnost použití pouze jednoho uzlu představuje nejslabší článek systému, který nemůže být použit pro řízení a vyčítání dat z větší sítě detektorů.
     \item \textbf{Modularita} - systém implementuje pouze komunikační protokol vyčítacího rozhraní \textit{ATLAS Pix} \ref{chap:detectors:readouts:atlaspix}. Přidání podpory nového vyčítacího rozhraní představuje významnou modifikaci architektury systému a pro nasazení nové verze je nutná odstávka celého systému.
 \end{enumerate}
Pro potřeby modernizace sítě \textit{ATLAS TPX} (za použití detektorů \textit{Timepix3}) bylo rozhodnuto o vývoji software, který bude navržen a implementován tak, aby požadavky na škálovatelnost a modularitu byly zajištěny.