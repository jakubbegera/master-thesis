\addbibresource{reference.bib}

\chapter{Testování}\label{chap:test}
V této kapitol bude čtenář seznámen s metodami testování pro zajištění kvality jednotlivých softwarových komponent a s provedenými experimenty.

%********************************************************************************
% Metody testování
%********************************************************************************
\section{Metody testování}
\subsection{Jednotkové testy}
Pro části jednotlivých softwarových komponent byly použity jednotkové testy \cite{testing_evans} (tzv. \textit{Unit testy}). Jednotkové testy jsou základním typem testu, který ověřuje funkčnost samostatně testovatelných částí systému - tzv. jednotek. Výhodou těchto testů je jejich automatizovatelnost. Byl použit \textit{JUnit} framework - framework pro jednotkové testy pro platformu Java. 

\subsection{Integrační testy}
Integrační testy byly použity pro ověření správné komunikace jednotlivých komponent systému, zejména pak handlera (viz \ref{chap:handler}) a backendové aplikace mastera (viz \ref{chap:master:backend}).

\subsection{Systémové testy}

%********************************************************************************
% Experimenty
%********************************************************************************
\section{Experimenty}