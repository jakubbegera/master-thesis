%\chapter{Závislosti modulu Handler}\label{chap:app:handler_dependencies}
\chapter{Přílohy vyčítacího rozhraní Katherine}\label{chap:app:katherine}

\section{Přehled DAC vyčítacího rozhraní Katherine}\label{chap:app:katherine:dacs}
\begin{table}[h]
	\begin{center}
		\begin{tabular}{|c|c|c|}
			\hline
            \textbf{Název} & \textbf{ID pro čtení} & \textbf{ID pro zápis} \\
			\hline
            Ibias\_Preamp\_ON & 1 & 0 \\
            Ibias\_Preamp\_OFF & 2 & 1 \\
            VPreamp\_NCAS & 3 & 2 \\
            Ibias\_Ikrum & 4 & 3 \\
            Vfbk & 5 & 4 \\
            Vthreshold\_fine & 6 & 5 \\
            Vthreshold\_coarse & 7 & 6 \\
            Ibias\_DiscS1\_ON & 8 & 7 \\
            Ibias\_DiscS1\_OFF & 9 & 8 \\
            Ibias\_DiscS2\_ON & 10 & 9 \\
            Ibias\_DiscS2\_OFF & 11 & 10 \\
            Ibias\_PixelDAC & 12 & 11 \\
            Ibias\_TPbufferIn & 13 & 12 \\
            Ibias\_TPbufferOut & 14 & 13 \\
            VTP\_coarse & 15 & 14 \\
            VTP\_fine & 16 & 15 \\
            Ibias\_CP\_PLL & 17 & 16 \\
            PLL\_Vcntrl & 18 & 17 \\
            BandGap output & 28 & - \\
            BandGap\_Temp & 29 & - \\
            Ibias\_dac & 30 & - \\
            Ibias\_dac\_cas & 31 & - \\
			\hline
		\end{tabular}
	\end{center}
	\caption{Přehled DAC (digitálně analogových převodníků) vyčítacího rozhraní katherine.}
	\label{tab:app:dacs}
\end{table}

\section{Přehled HW příkazů vyčítacího rozhraní Katherine}\label{chap:app:katherine:hw_commands}
\begin{table}[h]
	\begin{center}
		\begin{tabular}{|c|l|}
			\hline
            \textbf{ID} & \textbf{Název} \\
			\hline
            0 & Sensor Config Registers Update \\
            1 & Internal DAC Update \\
            2 & Internal DAC Back Read \\
            3 & Timer Read \\
            4 & Timer Set \\
            5 & Reset Matrix Sequential \\
            6 & Stop Matrix Command \\
            7 & Load Column Test Pulse Register \\
            8 & Read Column Test Pulse Register \\
            9 & Load Pixel Register Configuration \\
            10 & Read Pixel Register Configuration \\
            11 & Read Pixel Matrix Sequential Setting \\
            12 & Read Pixel Matrix Data-Driven Setting \\
            13 & Chip ID Read \\
            14 & Output Block Config Update \\
            15 & Digital Test \\
			\hline
		\end{tabular}
	\end{center}
	\caption{Přehled HW příkazů vyčítacího rozhraní katherine.}
	\label{tab:app:hw_commands}
\end{table}

\section{Přehled registrů detektoru Timepix3 v rámci vyčítacího rozhraní Katherine}\label{chap:app:katherine:tpx3_registers}
\begin{table}[h!]
	\begin{center}
		\begin{tabular}{|c|l|}
			\hline
            \textbf{ID} & \textbf{Název} \\
			\hline
            0 & Test Pulse Period \\
            1 & Number Test Pulses \\
            2 & Out Block Config \\
            3 & PLL Config \\
            4 & General Config \\
            5 & SLVS Config \\
            6 & Power Pulsing Pattern \\
            7 & SetTimer 15..0 \\
            8 & SetTimer 31..16 \\
            9 & SetTimer 47..32 \\
            10 & Sense DAC Selector \\
            11 & Ext DAC Selector \\
			\hline
		\end{tabular}
	\end{center}
	\caption{Přehled registrů detektoru Timepix3 v rámci vyčítacího rozhraní Katherine.}
\end{table}

\section{Konfigurace pixelů detektoru}\label{chap:app:katherine:pix_config}
\begin{code}[h!]
\begin{minted}[
frame=single,
linenos,
breaklines
]{Kotlin}
fun bmcToMatrixConfig(bmc: ByteArray): ByteArray {
    assert(bmc.size == 65536)
    val buff = IntArray(16384)

    for (i in bmc.indices) {
        var y = i / 256
        val x = i % 256
        val tmp = bmc[i]
        y = 255 - y
        buff[64 * x + (y shr 2)] = buff[64 * x + (y shr 2)] or (tmp.toInt() shl 8 * (3 - y % 4))
    }

    val out = ByteArray(65536)
    var k = 0
    for (mem in buff) {
        val b = ByteBuffer.allocate(4).putInt(mem).array()
        for (i in 0..3) {
            out[k + 3 - i] = b[i]
        }
        k += 4
    }

    return out
}
\end{minted}
\caption{Ukázka kódu pro převod pixelové konfigurace detektoru z formátu \texttt{BMC} do formátu podporovaného vyčítacím rozhraním \textit{Katherine}.}
\end{code}